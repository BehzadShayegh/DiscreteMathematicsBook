\newpage
\thispagestyle{empty}
\begin{center}
    مشخصات کتاب
\end{center}

\rule{\textwidth}{0.5pt}

\scriptsize{

نام: ریاضیات گسسته

نویسنده:
\begin{AFFILIATIONS}
    \AFFILIATIONROW
    {بهزاد شایق بروجنی}{دانشجوی کارشناسی مهندسی کامپیوتر دانشگاه تهران}{\AFFILIATIONURL{behzadshayegh.github.io}{https://behzadshayegh.github.io/}}
    {سودابه محمدهاشمی}{دانشجوی کارشناسی مهندسی کامپیوتر دانشگاه تهران}{\AFFILIATIONURL{soudabe.mhashemi@ut.ac.ir}{mailto:soudabe.mhashemi@ut.ac.ir}}
    {کیمیا محمدطاهری}{دانشجوی کارشناسی مهندسی کامپیوتر دانشگاه تهران}{\AFFILIATIONURL{k.m.taheri@ut.ac.ir}{mailto:k.m.taheri@ut.ac.ir}}
\end{AFFILIATIONS}

استاد ناظر:
\begin{AFFILIATIONS}
    \AFFILIATIONROW
    {دکتر سیامک محمدی}{عضو هیئت علمی دانشکده برق و کامپیوتر دانشگاه تهران}{\AFFILIATIONURL{ece.ut.ac.ir/~smohamadi}{https://ece.ut.ac.ir/~smohamadi}}
    {}{}{}
    {}{}{}
\end{AFFILIATIONS}

ویراستار:

صفحه‌آرا:
بهزاد شایق بروجنی

}

\rule{\textwidth}{0.5pt}

\scriptsize{
\p
این اثر با هدف کمک به تحصیل رایگان، به صورت آزاد در دسترس عموم قرار گرفته است.
هرگونه استفاده شخصی، چاپ و انتشار رایگان آن آزاد می‌باشد اما
با توجه به هدف این اثر، درآمدزایی از آن غیراخلاقی بوده و صاحبان اثر
به این امر راضی نخواهند بود.
\p
این اثر به صورت متن‌باز درحال توسعه است. هر داوطلبی می‌تواند
با برقراری ارتباط با مجموعه، در بهبود و تکمیل آن، به هر نحوی سهیم باشد.
این کار نیک، اثری ماندگار بر تحصیل آزاد خواهد داشت و لطف
حامیان و دست‌اندرکاران آن هرگز فراموش نخواهد شد.
\p
برای بازدید از منبع کتاب می‌توانید به صفحه
گیتهاب\LTRfootnote{\href{https://github.com/BehzadShayegh/DiscreteMathematicsBook}{github.com/BehzadShayegh/DiscreteMathematicsBook}}
این اثر مراجعه کنید.
همچنین فایل قابل دانلود اثر نیز در
همین صفحه\LTRfootnote{\href{https://github.com/BehzadShayegh/DiscreteMathematicsBook/tree/master/PublishedVersions}{github.com/BehzadShayegh/DiscreteMathematicsBook/tree/master/PublishedVersions}}
دردسترس است.
درصورت تمایل برای سهیم شدن در این جریان،
موقتاً از طریق
ایمیل\LTRfootnote{\href{mailto:behzad.shayegh@ut.ac.ir}{behzad.shayegh@ut.ac.ir}}
با ما در ارتباط باشید.
با ذکر نام، از زحمات شما قدردانی خواهد شد.
}
