\EXERCISE
 تعداد $n$ اتومبیل  بر روی محیط دایره $C$ خاموش ایستاده‌اند. فرض کنید ماشین $i$ که $1 \leq i \leq n$
    ، دارای 
    $L_{i}$
     لیتر بنزین است. اگر بدانیم که 
    $L = \sum L_{i}$
    آنگاه $L$ لیتر بنزین، سوخت لازم برای پیمایش دقیقا محیط $C$ می‌باشد. اثبات کنید برای هر مقدار 
$n \geq 1$
    ، اتومبیلی  وجود دارد که می‌تواند روی محیط دایره‌ی $C$ را یک دور بزند و به جایگاه اولیه‌اش بازگردد. در نظر بگیرید که هر اتومبیل  بتواند سوخت اتومبیل  دیگری
    را دریافت کند به شرطی که در مسیر، به آن اتومبیل  رسیده باشد.