\EXERCISE
تابع
$\varphi$
اویلر روی اعداد طبیعی
$n$
به شکل زیر تعریف می‌شود.
$\varphi(n)$
= تعداد اعداد طبیعی کوچک‌تر یا مساوی
$n$
که نسبت به آن اولند. نشان دهید:
$$\varphi(n) = n \Pi_{i=1}^{m}(1 - \frac{1}{pi})$$
که در آن
$p_1, p_2, \cdots, p_m$
تمام عوامل و متمایز
$n$
می‌باشند. از اصل شمول و عدم شمول استفاده کنید.