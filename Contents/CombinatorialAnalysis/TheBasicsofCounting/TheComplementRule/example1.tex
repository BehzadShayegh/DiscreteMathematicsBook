\begin{PROBLEM}
    \p
    در هر یک از اشکال زیر، چند طریق سفر از نقطه پایین سمت چپ تصویر به نقطه بالا سمت راست وجود دارد، اگر در هر نقطه
    فقط مجاز به حرکت به سمت راست یا بالا باشیم؟
    می‌دانیم تعداد مسیر‌های متمایز برای سفر از نقطه پایین سمت چپ یک جدول کامل (بدون حفره) با ابعاد 
    $n \times m$
    به نقطه‌ی بالا سمت راست آن، مطابق با شرایط مسئله، برابر 
    $\frac{(n+m)!}{n!m!}$
    می‌باشد.

    \begin{enumerate}
        \item
        حالتی که تنها یک حفره وجود داشته باشد: 
        \p
        \centerimage{0.35}{./1.jpg}

        \SOLUTION{
            \p
            برای پاسخ به این سوال، از اصل متمم استفاده می‌کنیم. با توجه به دانسته‌ی صورت سوال، بدون درنظر گرفتن حفره‌ی موجود و کامل فرض کردن
            جدول، تعداد مسیر‌های ممکن برابر است با:
            $$\frac{10!}{5!5!} = 252$$
            \p
            حال تلاش می‌کنیم تعداد حالات نامطلوب را پیدا کنیم. برای این منظور، باید تعداد مسیر‌هایی را بیابیم که از نقطه‌ی
            مفقود عبور می‌کنند. مطابق با اصل ضرب، تعداد این مسیر‌ها برابر است با حاصل ضرب تعداد مسیر‌های
            شروع شوند از نقطه ابتدایی و منتهی به نقطه‌ی مفقود، در تعداد مسیر‌های شروع شوند از نقطه مفقود و
            منتهی به نقطه‌ی پایانی.
            با توجه به دانسته‌ی سوال، تعداد مسیر‌های از نقطه‌ی ابتدایی تا نقطه‌ی مفقود برابر
            $$\frac{4!}{2!2!}$$
            و تعداد مسیر‌های از نقطه‌ی مفقود تا نقطه‌ی پایانی برابر
            $$\frac{6!}{3!3!}$$
            است.
            بنابراین، تعداد پاسخ‌های نامطلوب مسئله برابر است با:
            $$\frac{4!}{2!2!} \times \frac{6!}{3!3!} = 6 \times 20 = 120$$
            \p
            طبق اصل متمم، پاسخ مسئله برابر است با:
            $$252 - 120 = 132$$
        }

        \item 
        حالتی که دو حفره موازی وجود داشته باشد: 
        \p
        \centerimage{0.35}{./2.jpg}

        \SOLUTION{
            \p
            منظور از دو حفره موازی آن است که مسیری وجود ندارد که از روی هر دو حفره بگذرد.
            پاسخ مسئله مشابه قسمت قبل است با این تفاوت که در این قسمت، هنگام شمارش حالات نامطلوب از اصل جمع استفاده می‌کنیم.
            حالات نامطلوب را به دو دسته تقسیم می‌کنیم: ۱-مسیر‌هایی که از روی حفره‌ی سمت راست عبور می‌کنند و ۲-مسیر‌هایی که از روی حفره‌ی سمت چپ عبور می‌کنند.
            با توجه به موازی بودن دو حفره، نیازی نیست که هنگام محاسبه‌ی تعداد مسیر‌های نامطلوب هر دسته، نگران حفره دیگر باشیم.
            مشابه قسمت قبل، پاسخ مسئله برابر است با:
            $$252 - (\frac{4!}{2!2!} \times \frac{6!}{3!3!} + \frac{6!}{5!1!} \times \frac{6!}{5!1!}) = 252 - (120 + 36) = 96$$
        }
    \end{enumerate}
\end{PROBLEM}