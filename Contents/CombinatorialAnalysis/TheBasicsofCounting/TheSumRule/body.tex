\SUBSECTION{اصل جمع}

\begin{DEFINITION}
    \p
    \FOCUSEDON{اصل جمع:}
    اگر بتوان فرایندی را از دو مسیر متمایز انجام داد که مسیر اول
    $n$
    حالت برای انجام شدن و مسیر دوم
    $m$
    حالت برای انجام شدن داشته باشد و بین این دو دسته حالات، حالت مشترکی وجود نداشته باشد،
    آنگاه
    $n + m$
    حالت برای انجام شدن فرایند وجود دارد.
\end{DEFINITION}
    
\begin{THEOREM}
    \p
    \FOCUSEDON{تعمیم اصل جمع:}
    اگر بتوان فرایندی را از 
    $k$
    مسیر متمایز انجام داد که مسیر 
    $i$ام،
    $n_i$
    حالت برای انجام شدن داشته باشد و بین حالات مسیر‌های متفاوت، حالت مشترکی وجود نداشته باشد،
    آنگاه
    $\sum\limits_{i=1}^n n_i$
    حالت برای انجام شدن فرایند وجود دارد.
\end{THEOREM}

\NOTE{به الزام استقلال حالات انجام کار از مسیر‌های متفاوت توجه کنید. اگر مسیر‌های متفاوت دارای حالات انجام
مشترک باشند، دیگر مجاز به استفاده از اصل جمع نبوده و باید از اصل شمول و عدم شمول استفاده کنیم.}

\subfile{./example1.tex}