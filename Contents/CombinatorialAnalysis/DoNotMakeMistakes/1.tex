\EPROBLEM
\p
سه مهره رخ متمایز و صفحه شطرنجی $8\times8$ داریم. به چند روش می‌توان این سه مهره را در سه خانه از این صفحه قرار داد به طوری که حداقل یک مهره وجود داشته باشد که توسط هیچ مهره‌ای تهدید نمی‌شود؟

\EWSOLUTION{
    \p
    سوال را با اصل متمم حل می‌کنیم:
    
    -  کل حالات:
    
        \[64\times63\times62\]

    -  حالات نامطلوب: حالاتی که همه رخ‌ها تهدید بشوند.
    \p
        رخ اول برای قرارگیری در صفحه شطرنجی 64 حالت دارد, حال چون رخ اول باید تهدید بشود رخ دوم را باید در سطر یا ستون رخ اول قرار بدهیم که ۱۴ حالت دارد. چون رخ سوم هم باید تهدید بشود باید در سطر یا ستون یکی از رخ‌ها باشد که در مجموع شامل ۶ خانه در سطر یا ستون مشترک دو رخ قبلی و ۱۴ خانه در سطرها یا ستون‌های غیر مشترک دو رخ است. پس کل حالت‌ها برابر است با:
        
        \[64\times14\times20\]

    -  حالات مطلوب: طبق اصل متمم برابر است با:

        \[64\times63\times62 - 64\times14\times20\]
}

\NOTE{در اینجا تمام حالات نامطلوب محاسبه نشده است؛ زیرا این امکان وجود دارد که رخ اول توسط رخ دوم تهدید نشود و این حالت در نظر گرفته نشده است.}
\NOTE{بهتر بود اشاره شود که به دلیل تمایز رخ‌ها چنین نتیجه‌ای گرفته شده است.}

\EWSOLUTION{
    -  کل حالات:

        \[64\times63\times62\]

    -  حالات نامطلوب: حالاتی که همه رخ‌ها تهدید بشوند.
    
        \[64\times7\times20\times2\]

    -  حالات مطلوب: طبق اصل متمم برابر است با:

        \[64\times63\times62 - 64\times14\times20\]
}

\NOTE{به دلیل نبود توضیحات کافی, تشخیص چرایی غلط بودن جواب نهایی ممکن نیست اما با بررسی دقیق مسئله، به پاسخ متفاوتی خواهیم رسید.}

\ESOLUTION{
    -  کل حالات: به دلیل تمایز رخ‌ها برابر است با:

        \[ P(64,3) = 64\times63\times62\]

    -  حالات نامطلوب: حالاتی که همه رخ‌ها تهدید شوند.
    دو حالت داریم:
    \begin{enumerate}
        \item
        رخ اول رخ دوم را تهدید کند:
        \p
        رخ اول برای قرارگیری در صفحه شطرنجی 64 حالت دارد, حال چون رخ اول باید توسط رخ دوم تهدید شود،
        باید رخ دوم را در سطر یا ستون رخ اول قرار بدهیم که ۱۴ حالت دارد.
        چون رخ سوم هم باید تهدید شود، باید در سطر یا ستون یکی از رخ‌ها باشد که در مجموع شامل ۶ خانه در سطر یا ستون مشترک دو رخ قبلی و ۱۴ خانه در سطرها یا ستون‌های غیر مشترک دو رخ است. پس کل حالت‌ها برابر است با:
        
        \[64\times14\times20\]

        \item
        رخ اول رخ دوم را تهدید نکند:
        \p
        رخ اول برای قرارگیری در صفحه شطرنجی 64 حالت دارد. حال چون رخ اول نباید توسط رخ دوم تهدید شود، رخ دوم را در خانه‌ای خارج از سطر یا ستون رخ اول قرار می‌دهیم که ۴۹ حالت دارد. حال رخ سوم باید هر دو رخ را تهدید کند پس باید در یکی از محل‌های تقاطع سطر و ستون رخ اول و رخ دوم قرار گیرد که دو حالت دارد. پس کل حالت‌ها برابر است با:
        
        \[64\times49\times2\]
        
    \end{enumerate}

    -  حالات مطلوب طبق اصل متمم برابر است با:

        \[64\times63\times62\ - (64\times14\times20\ + 64\times49\times2\ )\]
}