\begin{PROBLEM}
    \p
    در شکل زیر که دارای دو حفره‌ی متوالی است، چند طریق سفر از نقطه پایین سمت چپ تصویر به نقطه بالا سمت راست وجود دارد، اگر در هر نقطه
    فقط مجاز به حرکت به سمت راست یا بالا باشیم؟
    \p
    \centerimage{0.35}{./1.jpg}

    \SOLUTION{
        \p
        لم اول: در یک جدول $n \times m$
        کامل (بدون حفره)،
        تعداد مسیر‌های ممکن برای سفر از نقطه پایین سمت چپ جدول
        به نقطه بالا سمت راست آن، درصورتی که فقط مجاز به حرکت بر روی خطوط و به سمت بالا یا راست باشیم،
        برابر است با:
        $$\frac{(n+m)!}{n!m!}$$
        \p
        اثبات لم اول:
        برای پیمودن مسیر مذکور، نیازمند به دقیقا ۵ حرکت رو به راست
        و ۵ حرکت رو به بالا هستیم.
        با توجه به آنکه حفره‌ای در مسیر وجود ندارد، ترتیب این حرکات می‌تواند هرچیزی باشد.
        مطابق با تعداد جایگشت‌های خطی با اعضای تکراری،
        تعداد جایگشت‌های این حرکات برابر است با:
        $$\frac{(n+m)!}{n!m!}$$
        \p
        با استفاده از لم فوق و به کمک اصل شمول و عدم شمول، پاسخ مسئله را بدست خواهیم آورد.
        با توجه به موازی بودن دو حفره، مسیر‌هایی وجود دارند که از روی هر دو حفره می‌گذرند.
        اگر مجموعه کل مسیر‌ها را $A$، مجموعه مسیر‌هایی که از روی حفره اول می‌گذرند را $B$
        و مجموعه مسیر‌هایی که از روی حفره دوم می‌گذرند را $C$ بنامیم، طبق اصل شمول و عدم شمول داریم:
        $$|B \cup C| = |B| + |C| - |B \cap C|$$
        و طبق اصل متمم، پاسخ مسئله برابر است با:
        $$|A| - |B \cup C|$$
        \p
        با استفاده از لم ذکر شده داریم:
        $$|A| = \frac{10!}{5!5!} = 252$$
        \p
        برای بدست آوردن اندازه مجموعه‌های $B$ و $C$، از اصل ضرب استفاده می‌کنیم.
        تعداد مسیر‌های مجموعه‌ی گذرنده از یک حفره، برابر حاصل ضرب تعداد کل مسیر‌های دو زیر جدول
        منتهی به آن حفره و آغاز شونده از آن حفره خواهد بود:
        $$|B| = \frac{4!}{2!2!} \times \frac{6!}{3!3!} = 120$$  
        $$|C| = \frac{7!}{4!3!} \times \frac{3!}{2!1!} = 105$$
        \p
        برای بدست آوردن اندازه‌ی مجموعه‌ی $B \cap C$ درست مانند قسمت قبل عمل می‌کنیم اما
        این بار مسیر را به سه قسمت (با تقاطع دو حفره) تقسیم می‌کنیم:
        $$|B \cap C| = \frac{4!}{2!2!} \times \frac{3!}{2!1!} \times \frac{3!}{2!1!} = 54$$
        \p
        همانطور که پیشتر نیز ذکر شد، پاسخ مسئله برابر است با:
        $$|A| - |B \cup C| = |A| - |B| - |C| + |B \cap C|$$
        $$= 252 - 120 - 105 + 54 = 81$$
    }

\end{PROBLEM}