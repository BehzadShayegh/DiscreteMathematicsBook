\SUBSECTION{ترکیب}

\begin{DEFINITION}
    \p
    هر انتخاب بدون ترتیب
    $r$
    عنصر از یک مجموعه
    $n$
    عضوی
    (یک زیرمجموعه $r$ عضوی از یک مجموعه $n$ عضوی)،
    یک
    \FOCUSEDON{$r$-ترکیب}
    از مجموعه
    $n$
    عضوی است.
    \FOCUSEDON{ترکیب}
    \FOCUSEDON{$r$}
    \FOCUSEDON{از}
    \FOCUSEDON{$n$}
    به معنای تعداد
    $r$-ترکیب‌های ممکن
    از یک مجموعه $n$ عضوی
    بوده که آن را با نماد 
    $C(n,r)$ یا 
    ${n\choose r}$ نمایش می‌دهیم. 
\end{DEFINITION}

\begin{THEOREM}
    \p
    اگر $n$ و $r$ اعدادی حسابی باشند به قسمی که 
    $r\leq n$، داریم:
    $${n \choose r} = \frac{n \times (n-1) \times ... \times (n-r+1)}{r!} = \frac{n!}{r!(n-r)!}$$
\end{THEOREM}

\p
رایج است که به جای عبارت
«ترکیب $r$ از $n$»
از عبارت 
«انتخاب $r$ از $n$»
استفاده شود.

\NOTE{
    دیدیم که یک
    $r$-ترکیب
    از مجموعه $A$،
    معادل یک زیرمجموعه $r$ عضوی از این مجموعه است.
    همچنین یک
    $r$-ترتیب
    از مجموعه $A$،
    جایگشتی خطی بر یک زیرمجموعه $r$ عضوی از این مجموعه است.
    مشخص است که یک $r$-ترتیب 
    حاصل محاسبه یک جایگشت خطی بر روی یک $r$-ترکیب است.
    بنابراین، می‌توان ترتیب را با کمک ترکیب تعریف کرد و طبق اصل ضرب می‌توانیم بنویسیم:
    $$P(n,r) = C(n,r)\times P(r,r)$$
    که $P(r,r)$ (مطابق انتظار) تعداد جایگشت‌های خطی مجموعه $r$ عضوی را محاسبه می‌کند.
}

\NOTE{
    واضح است که:
    $${n \choose r} = {n \choose n-r}$$
}

\subfile{./example1/body.tex}
\subfile{./example2/body.tex}

\begin{PROBLEM}[ترکیب چندگانه]
    \p
    تعداد افراز‌های ممکن یک مجموعه
    $n$
    عضوی به
    $k$
    زیرمجموعه‌ی نام‌دار (متمایز) به نحوی که اندازه مجموعه
    $i$ام
    برابر $n_{i}$
    باشد را بیابید.

    \SOLUTION{
        \p
        با توجه به تعریف افراز، داریم:
        $$n_{1} + n_{2} + ... + n_{k} = n$$
        \p
        مسئله را به $k$ گام تقسیم می‌کنیم
        و در گام
        $i$ام،
        تعداد حالات ساخت زیرمجموعه
        $n_i$ام
        را پیدا می‌کنیم.
        با توجه به ثابت بودن اندازه تمام زیرمجموعه‌ها،
        این تعداد برابر است با:
        $${n - \sum\limits_{j=1}^{i-1} n_j \choose n_i}$$
        طبق اصل ضرب، پاسخ مسئله برابر است با:
        $${n \choose n_{1}} \times {n-n_{1} \choose n_{2}} \times ... \times {n_{k} \choose n_{k}}$$
        $$= \frac{n!}{n_{1}!(n-n_{1})!} \times \frac{(n-n_{1})!}{n_{2}!((n-n_{1})!-n_{2})!} \times ... \times \frac{n_{k}!}{n_{k}!1!}$$
        $$= \frac{n!}{n_{1}!n_{2}! ... n_{k}!}$$
    }
\end{PROBLEM}

\begin{THEOREM}
    \TARGET{ترکیب چندگانه}
    \p
    تعداد افراز‌های ممکن یک مجموعه
    $n$
    عضوی به
    $k$
    زیرمجموعه‌ی نام‌دار (متمایز) به نحوی که اندازه مجموعه
    $i$ام
    برابر $n_{i}$
    باشد، برابر است با:
    $$\frac{n!}{n_{1}!n_{2}! ... n_{k}!}$$
\end{THEOREM}

\begin{DEFINITION}
    \p
    برای سادگی، نوشتار زیر را تعریف می‌کنیم:
    $${n \choose n_{1},n_{2}, ... n_{k}} = {n \choose n_{1}} \times {n-n_{1} \choose n_{2}} \times ... \times {n_{k} \choose n_{k}}$$
    $$= \frac{n!}{n_{1}!n_{2}! ... n_{k}!}$$
    و می‌خوانیم
    «ترکیب $n_1$ و $n_2$ و ... و $n_k$ از $n$».
    از این پس، در این کتاب، از عبارت بالا با نام
    \FOCUSEDON{ترکیب}
    \FOCUSEDON{چندگانه}
    یاد می‌کنیم.
\end{DEFINITION}

\subfile{./example3/body.tex}