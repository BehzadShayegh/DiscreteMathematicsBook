\SUBSECTION{معادلات سیاله خطی}
\TARGET{حل معادلات سیاله خطی به کمک توابع مولد}

\p
\CROSSREF[پیش‌تر]{تعداد پاسخ‌های معادله سیاله خطی با ضرایب واحد در مجموعه اعداد حسابی}
توانستیم تعداد پاسخ‌های معادله سیاله خطی با ضرایب واحد در مجموعه اعداد حسابی را به روش
توزیع اشیا محاسبه کنیم. این کار به کمک توابع مولد نیز ممکن بوده و در مواردی که معادله‌ی سیاله
مفروضات زیادی داشته باشد، عموما حل معادلات را ساده‌تر نیز می‌کند.

\subfile{./example1.tex}

\p
به کمک توابع مولد، یافتن تعداد پاسخ‌های معادله سیاله خطی به صورت کلی (بدون فرضی برای واحد بودن ضرایب) نیز در مجموعه اعداد حسابی ممکن است.
این روش را با حل یک مثال نشان می‌دهیم.

\begin{DEFINITION}
    \p
    به هر معادله‌ی چندجمله‌ای\footnote{حاصل جمع توان‌هایی از متغیر‌ها؛ در معادلات چندجمله‌ای،
    توابعی نمایی، لگاریتمی، سینوسی یا ... از متغیر‌ها مجاز نیستند.}
    خطی\footnote{توان تمام مجهولات واحد است.}
    که در آن متغیر‌ها فقط مجاز به اخذ مقادیر صحیح باشند،
    یک
    \FOCUSEDON{معادله}
    \FOCUSEDON{سیاله}\LTRfootnote{Diophantine Equation}
    \FOCUSEDON{خطی}\LTRfootnote{Linear Diophantine Equation}
    گفته می‌شود.
    شکل کلی این معادلات را می‌توان به صورت زیر نمایش داد:
    $$\sum\limits_{i}^{n} {a_i} {x_i} = s$$
\end{DEFINITION}

\subfile{./example2.tex}

\NOTE{
    به نحوه اعمال محدودیت‌ها در مثال‌های قبل توجه کنید. این کار با تغییر محدوده توان عبارات در
    تابع مولد انجام می‌شود. شاید متوجه شده باشید که با فاکتور گرفتن توان‌های مشترک،
    در مسیری مشابه روش اعمال محدودیت‌های معادله سیاله در حل به روش توزیع اشیا قرار می‌گیریم.
    به عنوان مثال در مسئله‌ای که ضریب
    $x^r$
    مطلوب است،
    اگر بتوانیم
    $x^k$
    را فاکتور بگیریم، مشخصا تنها لازم است ضریب عبارت
    $x^{r-k}$
    را در عبارت باقیمانده از فاکتورگیری محاسبه کنیم.
    همچنین در عبارت باقیمانده، شروط از بین رفته‌اند.
    به شباهت این تعبیر به تعبیر
    \CROSSREF[روش بیان شده در توزیع اشیا]{محدودیت حداقلی متغیر‌ها در معادله سیاله خطی با ضرایب واحد}
    توجه کنید.
    مقایسه‌ی دو پاسخ ارائه شده برای سوال بعد، این موضوع را بهتر نشان می‌دهد.
}

\subfile{./example3.tex}