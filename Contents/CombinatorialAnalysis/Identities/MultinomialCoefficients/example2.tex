\begin{PROBLEM}
  \p
  اثبات کنید:
  \begin{enumerate}
    \item 
    اگر 
    $n$
    عددی صحیح و نامنفی باشد:
      $$\sum\limits_{k=0}^n \binom{n}{k} = 2^n$$

    \SOLUTION{
      \p
      با توجه به برقرار بودن شرط قضیه ضرایب دوجمله‌ای، از آن استفاده می‌کنیم:
        $$2^n = (1+1)^n = \sum\limits_{k=0}^{n} \binom{n}{k} 1^k 1^{n-k} = \sum\limits_{k=0}^{n} \binom{k}{n}$$
    }

    \item 
    اگر
    $n$
    عددی صحیح و مثبت باشد:
      $$\sum\limits_{k=0}^n (-1)^k \binom{n}{k} = 0$$

    \SOLUTION{
      \p
      با توجه به برقرار بودن شرط قضیه ضرایب دوجمله‌ای، از آن استفاده می‌کنیم:
        $$0 = 0^n = ((-1)+1)^n = \sum\limits_{k=0}^{n} \binom{n}{k} (-1)^k 1^{n-k} = \sum\limits_{k=0}^{n} \binom{n}{k} (-1)^k$$
    }

    \item 
    اگر
    $n$
    عددی صحیح و مثبت باشد:
      $$\sum\limits_{k=0}^n \binom{2n}{2k} = \sum\limits_{k=1}^n \binom{2n}{2k-1}$$

    \SOLUTION{
      \p
      با توجه به برقرار بودن شرط حکم قسمت دوم همین سوال، از آن استفاده می‌کنیم:
        $$\sum\limits_{k=0}^{2n} \binom{2n}{k} (-1)^k = 0$$
        $$\sum\limits_{k=0}^{n} \binom{2n}{2k} (-1)^k + \sum\limits_{k=1}^{n} \binom{2n}{2k-1} (-1)^k = 0$$
        $$\sum\limits_{k=0}^{n} \binom{2n}{2k} + \sum\limits_{k=1}^{n} \binom{2n}{2k-1} (-1) = 0$$
        $$\sum\limits_{k=0}^{n} \binom{2n}{2k} - \sum\limits_{k=1}^{n} \binom{2n}{2k-1} = 0$$
        $$\sum\limits_{k=0}^{n} \binom{2n}{2k} = \sum\limits_{k=1}^{n} \binom{2n}{2k-1}$$
    }

    \item 
    اگر
    $n$
    عددی صحیح و نامنفی باشد:
      $$\sum\limits_{k=0}^n 2^k \binom{n}{k} = 3^n$$

    \SOLUTION{
      \p
      با توجه به برقرار بودن شرط قضیه ضرایب دوجمله‌ای، از آن استفاده می‌کنیم:
        $$3^n = (1+2)^n = \sum\limits_{k=0}^n \binom{n}{k} 1^{n-k} 2^k = \sum\limits_{k=0}^n \binom{n}{k} 2^k$$
    }

  \end{enumerate}
\end{PROBLEM}
