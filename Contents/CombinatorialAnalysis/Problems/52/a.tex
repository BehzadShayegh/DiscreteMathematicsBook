-
شکل
:A
\p
می‌دانیم اعداد 1 تا 6 را به !5 روش می‌توان دور شش ضلعی چید. فرض کنید شش ضلعی اولیه روی محور x و y قرار داشته باشد، حال اگر شکل را حول محور x در فضا 180 درجه بچرخانیم، حالت جدیدی بوجود نمی آید پس شکل از 2 جهت تقارن دارد و به
$\frac{5!}{2}$
حالت می‌توان آن را برچسب‌گذاری کرد.

-
شکل
:B
\p
دقیقا مانند استدلال شکل قبل
$\frac{5!}{2}$
حالت داریم.

-
شکل
:C
\p
دایره وسط را به 6 حالت میتوان برچسب‌گذاری کرد و دایره‌های کناری همگی در فضا متقارن‌اند پس جواب این حالت 6 است.

-
شکل 
:D
\p
این شکل نیز در فضا تقارن از 2 طرف تقارن دارد پس به 
$\frac{6!}{2}$
حالت می‌توان آن را برچسب‌گذاری کرد.	