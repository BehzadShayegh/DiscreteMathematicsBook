\p
حروف را به شکلی می‌شماریم تا کلمه‌ای پالیندروم با
$11$
حرف موجود بسازیم. باید حرفی که به تعداد فرد از آن در کلمه موجود است را وسط قرار ‌دهیم:
$$\bigcirc\bigcirc\bigcirc\bigcirc\bigcirc M\bigcirc\bigcirc\bigcirc\bigcirc\bigcirc$$
حال سمت راست یا چپ آن حرف را طوری پر می‌کنیم که بتوانیم سمت دیگر حرف وسط را با عکس آن رشته پر کنیم. پس لازم است نیمی از تعداد هر حرف را برای پر کردن نیمی از رشته استفاده کنیم. تعداد حالات شمارش حروف یک سمت با توجه به جایگشت با حروف تکراری، به صورت زیر است:
$$\frac{5!}{2!2!} = 30$$
به طور مثال، اگر رشته‌ی حروف
$IPSSI$
را انتخاب کنیم، داریم:
$$IPSSIM\bigcirc\bigcirc\bigcirc\bigcirc\bigcirc$$
\p
حال باید عکس آن، یعنی رشته‌ی حروف
$ISSPI$
را در سمت دیگر قرار دهیم:
$$IPSSIMISSPI$$
در حالت کلی نیاز به شمارش حالات سمت دیگر نداریم، چرا که کل کلمه زمانی که رشته‌ی حروف یک سمت انتخاب شده باشد، ثابت شده است. بنابراین تعداد کل حالات برابر با تعداد انتخاب‌های یک سمت، یعنی
$30$
می‌باشد.