\p
تعداد حالاتی که متحرک به مقصد می‌رسد و از نقطه 
$(7, 3)$ 
عبور می‌کند را محاسبه می‌کنیم، سپس تعداد حالاتی که متحرک هم از نقطه
$(7, 3)$ 
و هم از نقطه 
$(13, 3)$ 
رد می‌شود را از آن کم می‌کنیم. برای سهولت در بیان، ابتدا حرکت‌ها را نام گذاری می‌کنیم:
$$(x ,y) \to (x + 1, y + 1) : A , (x, y) \to (x + 1, y - 1) : B$$
   حالات رسیدن از مبدا به نقطه
$(17, 5)$
     با شرط آن که از نقطه
$(7, 3)$
      عبور کنیم را حساب کرده و در حالات رسیدن از آن نقطه به مقصد ضرب می‌کنیم:
    \begin{enumerate}
        \item 
        ابتدا با ترتیب دلخواه 
$5$ 
         بار حرکت 
$A$
          و
$2$ 
          بار حرکت 
$B$
           را انجام می‌دهیم تا از مبدا به نقطه 
$(7,3)$
           برسیم:
$$\frac{7!}{5!2!}$$
        \item
        سپس برای رسیدن از 
$(7, 3)$
         به 
$(17,5)$ 
        باید با ترتیب دلخواه
$6$ 
        بار حرکت 
$A$ 
        و 
$4$ 
        بار حرکت
$B$ 
        انجام شود:
$$\frac{10!}{4!6!}$$
    \end{enumerate}
    عدد به دست آمده برای رفتن از مبدا به مقصد با شرط عبور از خانه
$(7, 3)$ 
     برابر است با:
$$\frac{7!}{5!2!}\times\frac{10!}{4!6!}$$
    حال برای به دست آوردن تعداد حالاتی که از عدد فوق باید کم شود، تعداد حالاتی که از مبدا به مقصد می‌توان رسید با شرط عبور از
$2$ 
    نقطه
$(7, 3)$ 
    و
$(13, 3)$ 
     را به دست می‌آوریم به این صورت که، ابتدا از مبدا به نقطه
$(7, 3)$ 
، سپس از آنجا به نقطه
$(13, 3)$ 
     و بعد از آنجا به مقصد می‌رویم. اعداد به‌دست آمده از هر مرحله را در هم ضرب کرده تا جواب مدنظر به‌دست آید:
    \begin{enumerate}
        \item 
        برای رفتن از مبدا تا 
$(7, 3)$
$5$
          بار حرکت
$A$ 
           و
$2$ 
            بار حرکت
$B$ 
            را با ترتیب دلخواه انجام می‌دهیم:
$$\frac{7!}{5!2!}$$
        \item
        برای رفتن از 
$(7, 3)$ 
        به 
$(13, 3)$
$3$
          بار حرکت 
$A$ 
          و
$3$
           بار حرکت
$B$
            را با ترتیب دلخواه انجام می‌دهیم:
$$\frac{6!}{3!3!}$$
        \item
        برای رفتن از
$(13, 3)$
         به مقصد نیز 
$3$ 
         بار حرکت
$A$
         و
$1$ 
          بار حرکت
$B$
            را با ترتیب دلخواه انجام می‌دهیم: 
$$\frac{4!}{3!1!}$$
    \end{enumerate}
    عدد به دست آمده برای رفتن از مبدا به مقصد با شرط عبور از نقاط 
$(7, 3)$ 
    و 
$(13, 3)$ 
    برابر است با:
$$\frac{7!}{5!2!}\times\frac{6!}{3!3!}\times\frac{4!}{3!1!}$$
    با کم کردن 
$2$ 
    عدد نهایی به دست آمده در فوق از هم، جواب نهایی به دست می‌آید:
    $$\frac{7!}{5!2!}\times\frac{10!}{4!6!} - \frac{7!}{5!2!}\times\frac{6!}{3!3!}\times\frac{4!}{3!1!}$$