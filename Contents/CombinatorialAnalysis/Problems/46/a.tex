\begin{enumerate}
    \item
        3تا یک را به 
	$3!$
	حالت می‌توان در سطر و ستون‌های مختلف چید
	(سطر اول 3 حالت، سطر دوم 2 حالت و سطر سوم 1 حالت).
	حال عدد دو را به 2 حالت می‌توان در سطر اول گذاشت. مانند شکل زیر:
	
	 \begin{center}
	\begin{tabular}{ |c|c|c| } 
     \hline
       & 1 & 2  \\ 
     \hline
       &   & 1 \\ 
     \hline
     1 &  &   \\ 
     \hline
    \end{tabular}
    \end{center}
    اگر دقت کنید از اینجا به بعد با شروع از سطر اول، می‌توان تمام خانه‌ها را به طور یکتا پر کرد مثلا جدول بالا به صورت زیر پر می‌شود.
    
    \begin{center}
	\begin{tabular}{ |c|c|c| } 
     \hline
      3 & 1 & 2 \\ 
     \hline
      2 & 3 & 1 \\ 
     \hline
      1 & 2 & 3 \\ 
     \hline
    \end{tabular}
    \end{center}
    پس جواب این بخش
    $3!$ 
    است.
    
    \item
    اگر خانه وسط جدول 1 باشد, دو 1 دیگر را به دو حالت می‌توان در سطر اول و سوم قرار داد. بقیه مساله مانند حالت قبل حل می‌شود پس جواب این بخش 
    $2!$ است.
    \end{enumerate}