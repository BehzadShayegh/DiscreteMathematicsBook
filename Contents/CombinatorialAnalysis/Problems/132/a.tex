\p
یک زیرمجموعه از این نوع در نظر می‌گیریم، مثل
$\{1, 4, 7, 9, 12, 18\}$
و اعضای آن را به صورت
$1 \leq 1 < 4 < 7 < 9 < 12 < 18 \leq 31$
می‌نویسیم. می‌بینیم که مجموعه نابرابری‌ها تفاضل‌های
$1 - 1 = 0$
،
$4 - 1 = 3$
،
$7 - 4 = 3$
،
$9 - 7 = 2$
،
$12 - 9 = 3$
،
$18 - 12 = 6$
و
$31 - 18 = 13$
را معین می‌کنندو مجموع این تفاضل‌ها برابر
$30$
است. این مجموع برای تمام چنین مجموعه‌هایی ثابت و همین مقدار
$30$
می‌باشد. پس می‌توان گفت تناظری یک‌به‌یک بین تعداد چنین زیرمجموعه‌هایی و تعداد جواب‌های صحیح معادله‌ی
$x_1 + x_2 + x_3 + x_4 + x_5 + x_6 + x_7 = 30$
با شروط
$x_1, x_7 \geq 0$
و
$x_2, x_3, x_4, x_5, x_6 \geq 2$
برقرار است. پاسخ، ضریب
$x^{30}$
است عبارت زیر:
$$f(x) = (1 + x + x^2 + x^3 + \cdots)(x^2 + x^3 + x^4 + \cdots)^5(1 + x + x^2 + x^3 + \cdots) =$$
$$x^{10}(1 - x)^{-7}$$
پس این ضریب همان ضریب
$x^{20}$
در
$(1 - x)^{-7}$
است که برابر است با:
$$\binom{-7}{20}(-1)^{20} = \binom{20 + 7 - 1}{20} = \binom{26}{20} = \binom{26}{6}$$