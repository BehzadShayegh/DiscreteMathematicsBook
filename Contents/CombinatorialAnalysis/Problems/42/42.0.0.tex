\p
برای هر رقم یک تابع مولد نمایی داریم. برای هر یک از ارقام
$1$
و 
$2$
که محدودیتی نداریم:
$$E_1(x) = E_2(x) = \sum\limits_{n=0}^{\infty}\frac{x^n}{n!} = e^x$$
از رقم
$0$
باید حتما زوج تا داشته باشیم. بنابراین تابع مولد آن به شکل زیر است:
$$E_0(x) = \sum\limits_{n=0}^{\infty}\frac{x^{2n}}{(2n)!}$$
داریم:
$$e^x = 1 + \frac{x}{1!} + \frac{x^2}{2!} + \frac{x^3}{3!} + \cdots$$
$$e^{-x} = 1 - \frac{x}{1!} + \frac{x^2}{2!} - \frac{x^3}{3!} + \cdots$$
$$\frac{e^x + e^{-x}}{2} = 1 + \frac{x^2}{2!} + \frac{x^4}{4!} + \cdots = E_0(x)$$
تابع مولد کلی برابر است با:
$$E(x) = E_0(x)E_1(x)E_2(x) = \frac{e^x + e^{-x}}{2}.(e^x).(e^{-x}) = \frac{e^{3x} + e^x}{2}$$
تابع مولد نهایی را بسط می‌دهیم تا ضریب 
$\frac{x^n}{n!}$
را در آن پیدا کنیم:
$$\frac{e^{3x} + e^x}{2} = $$
$$\frac{1}{2}(\sum\limits_{n=0}^{\infty}\frac{(3x)^n}{n!} + \sum\limits_{n=0}^{\infty}\frac{x^n}{n!}) = $$
$$\frac{1}{2}(\sum\limits_{n=0}^{\infty}(3^n + 1)\frac{x^n}{n!}$$
بنابراین جواب نهایی برابر است با:
$$\frac{3^n+1}{2}$$

