    \p
    هدف از دو بخش اول این سوال آشنا کردن شما با تابع مولد معروف دنباله‌ی 
        $${1,1,1,1,1,1,\ldots}$$
        می‌باشد.
        تابع مولد این دنباله $\frac{1}{1-x}$ است. اثبات این رابطه نیز با استفاده از رابطه زیر صورت می‌گیرد:
        $$1 = (1-x)(1+x+x^2+x^3+\ldots) \rightarrow \frac{1}{1-x} = 1+x+x^2+x^3+\ldots$$
        
        که عبارت نهایی تابع مولد دنباله‌ی 
        $${1,1,1,1,1,1,\ldots}$$
        است.
        حال که این موضوع را می‌دانیم به حل سوال می‌پردازیم.
        \begin{enumerate}
            \item 
            برای حل این قسمت باید به سادگی ضرایب جمله‌های 5 و 6 و 7 را از دنباله عادی
            $${1,1,1,1,1,\ldots}$$
            کم کنیم. داریم:
            $$S_{1}(n) = \frac{1}{1-x} - x^5 - x^6 - x^7$$
            \item
            برای حل این قسمت هم مانند قسمت الف عمل می‌کنیم با این تفاوت که در این قسمت جمله‌های 3 و 4 و 5 و 6 و 7 با دنباله عادی متفاوت هستند. پس برای اینکه ضرایب را با دنباله مد نظر سوال یکی کنیم به صورت زیر عمل می‌کنیم:
            $$S_{2}(n) = \frac{1}{1-x} + 3x^3 + 2x^4 + 4x^5 + 5x^6 + x^7$$
        \item
        در این قسمت با استفاده از آموخته‌های خود از دو قسمت قبل و هم‌چنین توابع مولد عمل می‌کنیم. ابتدا برای هر کدام از شروط تابع مولد می‌نویسیم و سپس تابع مولد نهایی ما برابر حاصل‌ضرب تمامی  توابع مولد  به دست‌آمده خواهد بود.
        طبق این توضیحات به بررسی هر یک از شروط موجود در صورت سوال می‌پردازیم.
        \begin{itemize}
            \item 
            تیله‌های آبی باید زوج باشند:
            $$S'_{1} = 1 + x^2 + x^4 + \ldots = \frac{1}{1-x^2}$$
            \item
            تیله‌های زرشکی باید مضرب 13 باشند:
            $$S'_{2} = 1 + x^{13} + x^{26} + \ldots = \frac{1}{1-x^{13}}$$
            \item
            تیله‌های سبز باید کوچک‌تر مساوی 7 باشند:
            $$S'_{3} = 1 + x + x^2 + \ldots + x^7 = \frac{1-x^8}{1-x}$$
            \item
            تیله‌های بنفش از سه بیشتر باشند:
            \begin{align*}
            S'_{4} &= x^4 + x^5 + x^6 + \ldots\\
            &= (1+x+x^2+x^3+x^4+\ldots) - (1 + x + x^2 + x^3)\\
            &= (\frac{1}{1-x}) - (\frac{1-x^4}{1-x}) = \frac{x^4}{1-x}
            \end{align*}
        \end{itemize}
        همان‌گونه که بیان شد، تابع 
        $S(n)$
        حاصل‌ضرب توابع مولد فوق می‌باشد. پس در نهایت خواهیم داشت:
        $$S(n) = \frac{x^4(1-x^8)}{(1-x)^2(1-x^{13})(1-x^{2})}$$
        \end{enumerate}