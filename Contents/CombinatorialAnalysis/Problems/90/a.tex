        \p
بازیکنان را به ترتیب با شماره‌های
$1$
تا
$2n$   
مشخص می‌کنیم. فرض کنید بازیکن شماره
$i$
روز اول
$x_i$
امتیاز و روز دوم
$y_i$
امتیاز کسب کرده باشد و در ضمن به ازای
$1 \leq i \leq k$
،
$x_i \geq y_i$
و به ازای
$k + 1 \leq i \leq 2n$
،
$y_i > x_i$
. در این صورت طبق فرض:
$$x_1 - y_1 \geq n, \cdots, x_k - y_k \geq n, y_{k+1} - x{k+1} \geq n, \cdots, y_{2n} - x_{2n} \geq n$$  
$(*)$
\\
از جمع این
$2n$
نابرابری نتیجه می‌گیریم:
$$(x_1 + \cdots + x_k) - (y_1 + \cdots + y_k) - (x_{k+1} + \cdots + x_{2n})$$
$$+(y_{k+1} + \cdots + y_{2n}) \geq 2n^2$$
$(**)$
\\
در مجموع
$x_1 + \cdots + x_k$
امتیاز بازی‌های بین بازیکنان
$1$
تا
$k$
در روز اول و همچنین امتیاز بازی‌هایی که این
$k$
نفر در روز اول از
$2n - k$
بازیکن دیگر گرفته‌اند محاسبه می‌شود. چون هر بازی یک امتیاز دارد، لذا:
$$\binom{k}{2} \leq x_1 + \cdots + x_k \leq \binom{k}{2} + k(2n - k)$$
به روش مشابه نتیجه می‌گیریم:
$$y_1 + \cdots +y_k \geq \binom{k}{2}$$
$$x_{k+1} + \cdots + x_{2n} \geq \binom{2n-k}{2}$$
$$y{k+1} + \cdots + y_{2n} \leq \binom{2n-k}{2} + k(2n - k)$$
از چهار نابرابری اخیر نتیجه می‌گیریم:
$$(x_1 + \cdots + x_k) - (y_1 + \cdots + y_k) - (x_{k+1} + \cdots + x_{2n}) + (y_{k+1} + \cdots + y_{2n})$$
$$\leq \binom{k}{2} + k(2n - k) - \binom{k}{2} - \binom{2n-k}{2} + \binom{2n-k}{2} + k(2n - k)$$
$$= 2k(2n - k) < 2n^2$$  
 از مقایسه‌ی این نابرابری با نابرابری (**) نتیجه می‌گیریم کلیه‌ی نابرابری‌ها در واقع برابری بوده‌اند، علی‌الخصوص کلیه‌ی نابرابری‌های (*) برابری هستند، لذا حکم نتیجه می‌شود.