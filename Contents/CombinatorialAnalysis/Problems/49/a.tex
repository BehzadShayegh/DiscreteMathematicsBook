\p
این مساله از سه بخش مجزا تشکیل شده است که جواب نهایی آن طبق اصل ضرب حاصل ضرب جواب هر بخش می‌باشد.

\p
در بخش اول به سادگی
به
$\binom{m}{r}$
روش $r$ حرف متمایز را انتخاب می کنیم.

\p
از آنجایی که متن در $t$ سطر نوشته شده و هر سطر مقدار طبیعی حرف دارد. با حل معادله سیاله زیر در اعداد طبیعی می‌توان تعداد روش‌های بخش دو را نیز محاسبه کرد.

$$r_1 + r_2 + \ldots + r_t = n$$

می دانیم معادله بالا (در اعداد طبیعی) 
$\binom{n - 1}{t - 1}$
جواب دارد.

\p
برای محاسبه بخش آخر از تابع مولد کمک می‌گیریم.
اگر یک حرف، زوج تکرار داشته باشد تابع مولد آن به شکل زیر است:
\[ \sum_{n=0}^{+\infty} \frac{x^{2n}}{(2n)!} = \frac{e^x + e^{-x}}{2} \]
اگر یک حرف، فرد تکرار داشته باشد، تابع مولد آن به شکل زیر است:
\[ \sum_{n=0}^{+\infty} \frac{x^{2n + 1}}{(2n + 1)!} = \frac{e^x - e^{-x}}{2} \]

\p
پس جواب نهایی این بخش نیز، ضریب
$\frac{x^n}{n!}$
در عبارت زیر می‌باشد.

\[ \binom{r}{s}\times\left({\frac{e^x + e^{-x}}{2}}\right)^s\times\left({\frac{e^x - e^{-x}}{2}}\right)^{r - s} \]