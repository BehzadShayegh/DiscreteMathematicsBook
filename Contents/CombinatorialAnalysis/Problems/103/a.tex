\p
مدلی ترکیباتی را در نظر می‌گیریم.
$2n$
دانش‌آموز،
$n$
دانش‌آموز کلاس اول و
$n$
دانش‌آموز کلاس دوم، و یک نفر مربی در نظر بگیرید. دانش‌آموزان کلاس اول را با
$a_1, a_2, \cdots, a_n$
و دانش‌آموزان کلاس دوم را با
$b_1, b_2, \cdots, b_n$
نشان می‌دهیم. به ازای
$1 \leq i \leq n$
، فرض کنید
$(a_i, b_j)$
جفت متشکل از
$a_i$
و
$b_i$
باشد. به این دانش‌آموزان
$n$
بلیط برای دیدن یک بازی فوتبال پرهیجان اختصاص داده‌اند. تعداد راه‌هایی را که می‌توان
$n$
دانش آموز را برای دیدن این بازی انتخاب کرد را حساب می‌کنیم. معلوم است که این تعداد برابر است با:
$$\binom{2n + 1}{n}$$
از طرف دیگر، می‌توانیم این تعداد را به روش زیر حساب کنیم. به ازای هر عدد صحیح و ثابت مانند
$k$
،
$1 \leq k \leq n$
،
$k$
جفت از
$n$
جفت دانش‌آموز انتخاب می‌کنیم و به هر جفت یک بلیط می‌دهیم.
$\binom{n}{k}2^k$
راه برای انتخاب
$k$
جفت و برگزیدن یک دانش‌آموز از هر جفت برای رفتن به تماشای مسابقه وجود دارد.
$n - k$
بلیط و
$n - k$
جفت از دانش‌آموزان باقی مانده‌اند.
$\lfloor \frac{n - k}{2} \rfloor$
جفت انتخاب می‌کنیم و به هر یک از این جفت‌ها دو بلیط می‌دهیم. تعداد راه‌های انجام این کار برابر است با:
$$\binom{n - k}{\lfloor \frac{n - k}{2} \rfloor}$$
تا این‌جا
$k + 2\lfloor \frac{n - k}{2} \rfloor$
بلیط را پخش کرده‌ایم. این عدد را
$S$
بنامید. اگر
$n  - k$
فرد باشد، آن وقت
$S = n - 1$
و آخرین بلیط را هم به مربی می‌دهیم؛ اگر
$n  - k$
زوج باشد، آن وقت
$S = n$
و همه‌ی بلیط‌ها را پخش کرده‌ایم. به سادگی معلوم می‌شود که اگر
$k$
همه‌ی مقدارهای از
$1$
تا
$n$
را اختیار کند، همه‌ی راه‌های پخش کردن
$n$
بلیط را به‌دست آورده‌ایم. بنابرین تعداد راه‌های رفتن
$n$
نفر به تماشای مسابقه باربر است با:
$$\sum_{k=0}^{n} 2^k \binom{n}{k} \binom{n - k}{\lfloor \frac{n-k}{2} \rfloor}$$
به این ترتیب:
$$\sum_{k=0}^{n} 2^k \binom{n}{k} \binom{n - k}{\lfloor \frac{n-k}{2} \rfloor}  = \binom{2n + 1}{n}$$

