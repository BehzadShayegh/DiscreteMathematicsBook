\SECTION{دنباله و اعداد فیبوناچی}
فیبوناچی، ریاضیدان ایتالیایی قرن ۱۳ ‌مسئله‌ی معروف رشد جمعیت خرگوش‌ها را در کتاب خود مطرح کرده است.
این مسئله را در مثال زیر بیان می‌کنیم.

\subfile{./rabbitsProblem.tex}

\NOTE{توجه کنید به تعداد جملاتی که در رابطه بازگشتی به عقب برمی‌گردیم به جملات پایه نیاز داریم.
در تعریف دنباله‌ فیبوناچی چون
$f_{n-1}$
    و
    $f_{n-2}$ 
    داريم دو جمله پایه
    $f_{1} = 1$
    و
    $f_{2} = 1$
    را تعریف کردیم.  }
    
\begin{DEFINITION}
    به اعدادی که در رابطه بازگشتی
  \[\begin{cases}
      f_{n}=f_{n-1} + f_{n-2} & n\geq 2 \\
      
      f_0=1 ,
      f_1 = 1
  \end{cases}
  \]
  صدق می کنند,
  \FOCUSEDON{  دنباله‌ فیبوناچی}
    می‌گوییم.
\end{DEFINITION}


\subfile{./example1.tex}

\begin{THEOREM}
    \p
    \FOCUSEDON{تابع مولد دنباله فیبوناچی}
    برابر است با:
    $$F(x) = \frac{1}{1 - x - x^2}$$
    \FOCUSEDON{فرمول صریح فیبوناچی}
    برابر است با:
    $$f_n = \frac{1}{\sqrt{5}}((\frac{-1 + \sqrt{5}}{2})^{n} - (\frac{1 - \sqrt{5}}{2})^{n})$$
\end{THEOREM}

\subfile{./example2.tex}


\begin{DEFINITION}
    \p
    \FOCUSEDON{دنباله فیبوناچی}
    دنباله‌ای از اعداد است که در آن به جز دو جمله اول، هر جمله از مجموع دو جمله‌ی قبلی به دست می‌آید.
    \p
  $$1, 1, 2, 3, 5, 8, 13, 21, 34, 55, 89, ...$$
	\p 
\end{DEFINITION}

\begin{EXTRA}{دنباله فیبوناچی}
  \p
دنباله فیبوناچی خواص شگفت‌انگیزی دارد که باعث می شود در آثار برجسته هنر و معماری، دانه‌های گل آفتاب گردان، صدف‌ها و تولید مثل خرگوش‌ها، آثار آن دیده شود. به زودی مطالب بیش‌تری در این قسمت قرار خواهد گرفت.
\end{EXTRA}