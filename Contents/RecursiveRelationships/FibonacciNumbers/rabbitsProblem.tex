\begin{PROBLEM}
\p
فیبوناچی علاقه‌مند بود بداند
 اگر در جزیره‌ای عجیب يک جفت خرگوش نر و ماده داشته باشیم که رفتاری عجیب برای زاد و ولد خود دارند پس از گذشت 
 $n$
 ماه چند خرگوش داریم.

-  يک جفت خرگوش نر و ماده داريم که همين الان بدنيا آمده‌اند.

- خرگوش‌ها پس از يک ماه بالغ می‌شوند.

- دوران بارداري خرگوش‌ها يک ماه است.

- هنگامي که خرگوش ماده به سن بلوغ مي رسد حتما باردار می‌شود.

- در هر بار بارداري خرگوش ماده يک خرگوش نر و يک ماده بدنيا می‌آورد.

- خرگوش‌ها هرگز نمی‌ميرند. 

\SOLUTION{
    \p
    فرض کنید 
    $f_{n}$
    تعداد جفت خرگوش‌ها در ماه
    $n$ام
    باشد.
    سعی می‌کنیم سوال را به صورت بازگشتی حل کنیم.
    در پایان ماه اول
    $f_{1} = 1$
    است.
     از آن ‌جایی که در این ماه زاد و ولدی صورت نمی‌گیرد پس در پایان ماه دوم هم 
     $1$
     جفت خرگوش خواهیم داشت.
     با همین استدلال در پایان ماه سوم 
     $2$
     جفت خرگوش داريم یعنی
      $f_{3} = 2$.
      با کمی دقت متوجه می‌شویم 
      $f_{n} = f_{n-1} + f_{n-2}$.
      طبق فرض سوال جملات پایه 
    $f_{1} = 1$
    و
    $f_{2} = 1$
    است.
}

\end{PROBLEM}
