\begin{PROBLEM}[زاد و ولد خرگوش‌ها]
\p
مردی 
 یک جفت خرگوش نر و ماده که رفتاری عجیب برای زاد و ولد خود دارند، در محیطی بسته نگه می‌دارد. او می‌خواهد بداند پس از گذشت 
 $n$
ماه چند خرگوش دارد.
تعداد خرگوش‌ها را به صورت رابطه بازگشتی بنویسید.

-  یک جفت خرگوش نر و ماده داریم که همین الان بدنیا آمده‌اند.

- خرگوش‌ها پس از یک ماه بالغ می‌شوند.

- دوران بارداری خرگوش‌ها یک ماه است.

- هنگامی که خرگوش ماده به سن بلوغ می‌رسد، حتما باردار می‌شود.

- در هر بار بارداری، خرگوش ماده یک خرگوش نر و یک ماده به دنیا می‌آورد.

- خرگوش‌ها هرگز نمی‌میرند. 

\SOLUTION{
    \p
    فرض کنید 
    $f_{n}$
    تعداد جفت خرگوش‌ها در ماه
    $n$ام
    باشد.
    سعی می‌کنیم سوال را به صورت بازگشتی حل کنیم.
    در پایان ماه اول
    $f_{1} = 1$
    است.
     از آن ‌جایی که در این ماه زاد و ولدی صورت نمی‌گیرد پس در پایان ماه دوم هم 
     $1$
     جفت خرگوش خواهیم داشت.
     با همین استدلال در پایان ماه سوم 
     $2$
     جفت خرگوش داریم یعنی
      $f_{3} = 2$.
      به طور کلی متوجه می‌شویم در پایان ماه  
      $n$
      ام به تعداد 
      $f_{n-1}$
      خرگوش‌ بالغ و نابالغ داریم که از بین این تعداد
      ،
      $f_{n-2}$
      خرگوش‌ بالغ هستند که بچه به دنیا می‌آورند پس
      $f_{n} = f_{n-1} + f_{n-2}$.
      طبق فرض سوال جملات پایه 
    $f_{1} = 1$
    و
    $f_{2} = 1$
    است.
}

\end{PROBLEM}
