\SUBSECTION{حل روابط بازگشتی به کمک توابع مولد}
\p
روابط بازگشتی را می‌توان به کمک توابع مولد حل کرد.
مراحل زیر را به ترتیب در نظر بگیرید.
\begin{enumerate}[label=\arabic*)]
    \item تابع مولدی را تعریف می‌کنیم.
    \item طرفین روابطه بازگشتی را در $x^n$ ضرب می‌کنیم.
    \item برای همه اعداد حسابی روابطه بازگشتی مرحله ۲ را می‌نویسیم و با هم جمع می‌کنیم.
    \item عبارات را تفکیک می‌کنیم سپس با استفاده از عملیات‌های فاکتورگیری و لغزش حدود روی سیگما، اندیس‌های روابطه بازگشتی و توان $x$ را در هر جمله یکسان می‌کنیم.
    \item جملات به دست آمده را برحسب تابع مولد مرحله ۱ می‌نویسیم.
    \item با ساده‌سازی عبارت‌ها تابع مولد را بر حسب تابعی از $x$ به دست می‌آوریم.
    \item عبارت‌های به دست آمده را بر حسب تابع مولد‌های پایه می‌نویسیم و ضریب $x^n$ را حساب می‌کنیم.
    \item ضریب $x^n$ پاسخ صریح روابطه بازگشتی است.
\end{enumerate}
نحوه این استفاده را با چند مثال نشان خواهیم داد.
سعی کنید مراحل بالا را در روند حل سوال دنبال کنید.

\subfile{./example1.tex}
\subfile{./example2.tex}

\p
برای به دست آوردن فرم بسته روابط بازگشتی، گاهی اوقات نیاز است از تابع مولد نمایی استفاده کنیم.

\subfile{./ExponentialExample1.tex}