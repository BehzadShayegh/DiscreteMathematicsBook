\SUBSECTION{حل روابط بازگشتی به کمک توابع مولد}
\p
برخی روابط بازگشتی را می‌توان به کمک توابع مولد حل کرد.
مراحل زیر را به ترتیب در نظر بگیرید.
\begin{enumerate}[label=\arabic*)]
    \item تابع مولد جملات دنباله بازگشتی مورد بحث را در نظر می‌گیریم.
    \item با هدف بازسازی تابع مولد مذکور،
     طرفین رابطه بازگشتی را در
    $x^n$
     ضرب می‌کنیم. سپس لازم است تساوی حاصل از محاسبه‌یِ مجموعِ جملاتِ طرفینِ رابطه‌ی به دست آمده به ازای تمام 
     $n$
     های حسابی محاسبه شود.
    \item عبارات را تفکیک و سپس با استفاده از عملیات‌های فاکتورگیری و لغزش حدود روی سیگما، اندیس‌های جملات رابطه بازگشتی و درجه $x$ را در هر جمله یکسان می‌کنیم.
    (دقت کنید هدف بازسازی تابع مولد است و در این رابطه اندیس‌ ضریب هر جمله برابر درجه $x$ در همان جمله است).
    \item جملات به دست آمده را برحسب تابع مولد مرحله ۱ می‌نویسیم.
    \item با ساده‌سازی عبارت‌ها تابع مولد را بر حسب تابعی از $x$ به دست می‌آوریم.
    \item عبارت‌های به دست آمده را بر حسب تابع مولد‌های پایه می‌نویسیم و ضریب $x^n$ را حساب می‌کنیم.
    \item ضریب $x^n$ پاسخ صریح رابطه بازگشتی است.
\end{enumerate}
\p
نحوه‌ی این استفاده را با چند مثال نشان خواهیم داد.
سعی کنید مراحل بالا را در روند حل سوال دنبال کنید.

\subfile{./example1.tex}
\subfile{./example2.tex}

\p
برای به دست آوردن فرم بسته روابط بازگشتی، گاهی اوقات نیاز است از تابع مولد نمایی استفاده کنیم.

\subfile{./ExponentialExample1.tex}