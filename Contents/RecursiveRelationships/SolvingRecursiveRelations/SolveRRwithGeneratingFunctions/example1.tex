\begin{PROBLEM}
    \p
    پاسخ صریحی برای حداقل تعداد حرکات لازم بازی هانوی را به کمک رابطه بازگشتی
    $h_n$
    بدست آورید.
    $n$
        تعداد دیسک‌های برج است.
        $$h_1 = 1, h_n = 2h_{n-1} + 1$$
        

    \SOLUTION{
        \p
        لازم است $h_0$ را از روی رابطه به دست می‌آوریم:
        
        $$h_1 = 1 = 2h_0 + 1 \Rightarrow h_0 = 0$$

        \begin{enumerate}[label=\arabic*)]
            \item
            \textbf{
                تابع مولد جملات دنباله بازگشتی مورد بحث را در نظر می‌گیریم.
            }
            \p
            تابع مولد $G(x)$ را به شکل زیر تعریف می‌کنیم:
            
            $$G(x) = \sum_{n = 0}^{\infty} h_nx^n$$

            \item
            \textbf{
                با هدف بازسازی تابع مولد مذکور در رابطه‌ی بازگشتی خود،
                طرفین این رابطه را در
                $x^n$
                ضرب می‌کنیم. سپس لازم است تساوی حاصل از محاسبه‌ی مجموعِ جملاتِ طرفینِ رابطه‌ی به دست آمده به ازای تمام 
                $n$
                های حسابی محاسبه شود.
            }

            \p
            از آنجایی که
            $ n - 1 \geq 0 \rightarrow n \geq 1 $
            پس حدود سیگما را از یک تا بی‌نهایت قرار می‌دهیم.
            
            $$\sum_{n = 1}^{\infty} h_nx^n = \sum_{n = 1}^{\infty} (2h_{n-1}+1)x^n$$

            \item
            \textbf{
                عبارات را تفکیک و سپس با استفاده از عملیات‌های فاکتورگیری و لغزش حدود روی سیگما، اندیس‌های جملات رابطه بازگشتی و درجه $x$ را در هر جمله یکسان می‌کنیم
                (دقت کنید هدف بازسازی تابع مولد است و در این رابطه اندیس‌ ضریب هر جمله برابر درجه $x$ در همان جمله است).
            }

            $$\sum_{n = 1}^{\infty} h_nx^n = 2\sum_{n = 1}^{\infty} h_{n-1}x^n + \sum_{n = 1}^{\infty}x^n$$
            $$\sum_{n = 0}^{\infty} h_nx^n - h_0= 2x\sum_{n = 1}^{\infty} h_{n-1}x^{n-1} + x\sum_{n = 0}^{\infty}x^n$$
            $$\sum_{n = 0}^{\infty} h_nx^n - h_0 = 2x\sum_{n = 0}^{\infty} h_{n}x^{n} + x\sum_{n = 0}^{\infty}x^n$$
            
            \item
            \textbf{
                جملات به دست آمده را برحسب تابع مولد مرحله ۱ یا تابع مولد‌های پایه می‌نویسیم.
            }
            
            $$G(x) - 0 = 2xG(x) + x\frac{1}{1 - x}$$
            
            \item
            \textbf{
                با ساده‌سازی عبارت‌ها تابع مولد را بر حسب تابعی از $x$ به دست می‌آوریم.
            }
            
            $$G(x) \times (1 - 2x) = \frac{x}{1 - x}$$
            $$G(x) = \frac{x}{(1 - x)(1 - 2x)}$$
            
            پس از تفکیک کسر داریم:
            
            $$G(x) = \frac{-1}{1 - x} + \frac{1}{1 - 2x}$$
            
            \item
            \textbf{
                عبارت‌های به دست آمده را بر حسب تابع مولد‌های پایه می‌نویسیم و ضریب $x^n$ را حساب می‌کنیم.
            }
            
            $$G(x) = -\sum_{n = 0}^{\infty} x^n + \sum_{n = 0}^{\infty} 2^nx^n$$
            $$G(x) = \sum_{n = 0}^{\infty} (2^n - 1)x^n$$
            
            \item
            \textbf{
                ضریب $x^n$ پاسخ صریح رابطه بازگشتی است.
            }
            \p
            طبق تعریفی که در ابتدای سوال از تابع مولد ارائه دادیم، ضریب $x^n$ همان $h_n$ است
            که در اینجا برابر
            $2^n - 1$
            می‌باشد.
        \end{enumerate}
    }
\end{PROBLEM}