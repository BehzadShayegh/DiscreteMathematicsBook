
\p
در این قسمت ابتدا با برخی روش‌های اولیه حل روابط بازگشتی آشنا می‌شویم.
دو روش ابتدایی برای حل روابط بازگشتی, روش‌های «حدس و استقرا» و «گام به گام» که آن‌ها را با حل یک مسئله بررسی می‌کنیم:


\begin{PROBLEM}
    \p
    جواب رابطه بازگشتی
    $$f(n)=f(n-1)+2n-1$$
    $$f(1)=1$$
    را بیابید.
    \SOLUTION[روش حدس و استقرا]{
        \NOTE[-1cm]{
            در این روش ابتدا با کارهایی مانند جایگذاری در رابطه, جواب رابطه را حدس می‌زنیم. سپس به کمک استقرا درستی حدس خود را اثبات می‌کنیم.
            }
        \p
        ابتدا با جایگذاری در رابطه، چند جمله اول آن را به دست می‌آوریم:
        $$f(1)=1$$
        $$f(2)=f(1)+3=4$$
        $$f(3)=f(2)+5=9$$
        $$f(4)=f(3)+7=16$$
        با توجه به اعداد به دست آمده, حدس می‌زنیم که جواب برابر 
        $f(n)=n^2$
        باشد.
        می‌خواهیم به کمک استقرا حدس خود را ثابت کنیم.
        \begin{itemize}
            \item پایه استقرا:
            $$f(1)=1^2=1$$
            \item فرض استقرا:
            $$f(n)=n^2$$
            \item حکم استقرا:
            $$f(n+1)=(n+1)^2$$
        \end{itemize}
        اثبات حکم استقرا:
        $$f(n+1)=f(n)+2(n+1)-1=n^2+2n+1=(n+1)^2$$
        بنابراین ثابت می‌شود که حکم استقرا صحیح است و جواب رابطه بازگشتی برابر

        $f(n)=n^2$
        است.
    }
    \SOLUTION[روش گام به گام]{
        \NOTE[-1cm]{
            می توان با جایگذاری مکرر 
            رابطه بازگشتی اولیه در 
              $f(i)$
              های سمت راست تساوی, به تساوی رسید که یک طرف آن 
              $f(n)$
              و طرف دیگر آن حاصل جمع تعدادی جمله بر حسب n
              است.
              عبارت سمت راست ممکن است برحسب مجموع جملات یک یا چند دنباله باشد که با محاسبه آن در فصل مجموعه‌ها آشنا شدیم, در این صورت می توان با محاسبه این مجموع به معادله صریح 
              $f(n)$
            رسید
        }
        \p
        $$f(n)=f(n-1)+2n-1=f(n-2)+2(n-1)-1+2n-1$$
        $$f(n)=\sum_{i=1}^{n} 2i-1=2\sum_{i=1}^{n}n+\sum_{i=1}^{n}1=2\times\dfrac{n(n+1)}{2}+n=n^2$$
    }
 \end{PROBLEM}
 