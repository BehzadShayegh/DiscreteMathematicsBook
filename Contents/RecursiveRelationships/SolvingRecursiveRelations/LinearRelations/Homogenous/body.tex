\SUBSECTION{حل روابط بازگشتی خطی همگن}

\begin{DEFINITION}
    \p
    اگر رابطه‌ بازگشتی خطی همگنی به فرم
    \[a_n=c_{1}a_{n-1}+c_{2}a_{n-2}+...+c_{k}a_{n-k}\]
    داشته باشیم, به معادله
    \[r^k-c_{1}r^{k-1}-c_{2}r^{k-2}-...-c_{k}=0\]
    \FOCUSEDON{معادله مشخصه}
    %\footnote{characteristic-equation}
     این رابطه‌ بازگشتی می‌گوییم.
     جواب‌های این معادله \FOCUSEDON{ریشه‌های مشخصه}
    %\footnote{(characteristic-roots)}
    نامیده می‌شوند.

\end{DEFINITION}
\begin{PROBLEM}
    \p
    نشان دهید
    $a_n=r^n$
    جواب یک رابطه‌ بازگشتی خطی همگن است, اگر و تنها اگر 
    $r$
    ریشه معادله مشخصه آن باشد.
    \SOLUTION{
        \p
        $a_n=r^n$
        جواب معادله بازگشتی است اگر و تنها اگر در رابط بازگشتی صدق بکند.
        \[a_n=c_{1}a_{n-1}+c_{2}a_{n-2}+...+c_{k}a_{n-k}\]
        \[r^n=c_{1}r^{n-1}+c_{2}r^{n-2}+...+c_{k}r^{n-k}\]
        از تقسیم طرفین بر
        $r^{n-k}$
        داریم:
        \[r^k=c_{1}r^{k-1}+c_{2}r^{k-2}+...+c_{k}\]
    }
\end{PROBLEM}
\p
حال
ابتدا به بیان روش حل معادله مشخصه درجه دوم می‌پردازیم.
\begin{THEOREM}
    \p
    رابطه‌ بازگشتی درجه دوم
    $a_n=c_1 a_{n-1}+c_2 a_{n-2}$
    را در نظر بگیرید, فرض کنید که معادله مشخصه آن دارای دو ریشه حقیقی و متمایز 
    $r_1$ و $r_2$
    است.
    در این صورت دنباله 
    $\{a_n\}$
    جواب مسئله است اگر و تنها اگر
    \[a_n=\alpha_1 r_1^n+\alpha_2 r_2^n\]
    که
    $\alpha_1$
    و
    $\alpha_2$
    اعداد ثابت هستند.

\end{THEOREM}
\begin{PROBLEM}
    \p
    جواب رابطه‌ بازگشتی 
    $a_n=a_{n-1}+2a_{n-2}$
    را در صورتی که
    $a_0=2$
    و
    $a_1=7$
    باشد به دست آورید.
    \SOLUTION{
        معادله مشخصه این رابطه‌
        $r^2-r-1=0$
        است که ریشه‌های آن
        $r_1=2$
        و
        $r_2=-1$
        است.
        طبق قضیه ۱, جواب
        $\{a_n\}$
        به فرم زیر است:
        \[a_n=\alpha_1 2^n+\alpha_2 (-1)^n\]
        که
        $\alpha_1$
        و
        $\alpha_2$
        اعداد ثابت هستند.
        حال از جایگذاری شرایط اولیه داریم:
        \[a_0=\alpha_1+\alpha_2=2\]
        \[a_1=2\alpha_1-\alpha_2=7\]
        \[\rightarrow \alpha_1=3\]
        \[\rightarrow \alpha_2=-1\]
        پس جواب نهایی برابر است با:
        \[a_n=3\times 2^n-(-1)^n\]
    }
\end{PROBLEM}
\begin{THEOREM}
    \p
    قضیه ۲:
    رابطه‌ بازگشتی درجه دوم
    $a_n=c_1 a_{n-1}+c_2 a_{n-2}$
    را در نظر بگیرید, فرض کنید که معادله مشخصه آن دارای یک ریشه حقیقی 
    $r_1$
    است.
    در این صورت دنباله 
    $\{a_n\}$
    جواب مسئله است اگر و تنها اگر
    \[a_n=\alpha_1 r_1^n+\alpha_2 nr_1^n\]
    که
    $\alpha_1$
    و
    $\alpha_2$
    اعداد ثابت هستند.

\end{THEOREM}
\begin{PROBLEM}
    \p
    جواب رابطه‌ بازگشتی 
    $a_n=6a_{n-1}-9a_{n-2}$
    را در صورتی که
    $a_0=1$
    و
    $a_1=6$
    باشد به دست آورید.
    \SOLUTION{
        \p
        معادله مشخصه این رابطه‌
        $r^2-6r+9=0$
        است که ریشه آن
        $r_1=3$
        است.
        طبق قضیه ۲, جواب
        $\{a_n\}$
        به فرم زیر است:
        \[a_n=\alpha_1 3^n+n\alpha_2 (3)^n\]
        که
        $\alpha_1$
        و
        $\alpha_2$
        اعداد ثابت هستند.
        حال از جایگذاری شرایط اولیه داریم:
        \[a_0=\alpha_1=1\]
        \[a_1=3\alpha_1+3\alpha_2=6\]
        \[\rightarrow \alpha_2=1\]
        پس جواب نهایی برابر است با:
        \[a_n=3^n+n3^n\]
    }
\end{PROBLEM}
\p
حال می‌توان مطالب مطرح شده را برای رابطه‌ بازگشتی درجه
$k$
تعمیم داد
\begin{THEOREM}
    \p
    قضیه ۳:
    رابطه‌ بازگشتی درجه 
    ام$k$
    \[a_n=c_{1}a_{n-1}+c_{2}a_{n-2}+...+c_{k}a_{n-k}\]
    را در نظر بگیرید, فرض کنید که معادله مشخصه آن دارای 
    $k$
    ریشه حقیقی و متمایز 
    $r_1, r_2, ...r_k $
    است.
    در این صورت دنباله 
    $\{a_n\}$
    جواب مسئله است اگر و تنها اگر
    \[a_n=\alpha_1 r_1^n+\alpha_2 r_2^n+...+\alpha_k r_k^n\]
    که
    $\alpha_1,\alpha_2,...\alpha_k$
    اعداد ثابت هستند.

\end{THEOREM}

\begin{PROBLEM}
    نمونه سوال اضافه شود
\end{PROBLEM}

\begin{THEOREM}
    \p
    قضیه ۴:
    رابطه‌ بازگشتی درجه 
    ام$k$
    \[a_n=c_{1}a_{n-1}+c_{2}a_{n-2}+...+c_{k}a_{n-k}\]
    را در نظر بگیرید, فرض کنید که معادله مشخصه آن دارای 
    $t$
    ریشه حقیقی و متمایز 
    $r_1, r_2, ...r_t $
    است که ریشه
    $i$ام 
    $m_i$
    بار تکرار شده است,  
    $m_i \geq 1$
    و
    $\sum_1^k m_i=k$
    در این صورت دنباله 
    $\{a_n\}$
    جواب مسئله است اگر و تنها اگر
    \[a_n=(\alpha_{1,0}+ n\alpha_{1,1}+...+n^{m_1-1}\alpha_{1,m_1-1})r_1^n\]
    \[+(\alpha_{2,0}+ n\alpha_{2,1}+...+n^{m_2-1}\alpha_{2,m_2-1})r_2^n\]
    \[+...\]
    \[+(\alpha_{t,0}+ n\alpha_{t,1}+...+n^{m_t-1}\alpha_{t,m_t-1})r_t^n\]
    \[=\sum_{i=1}^t\sum_{j=0}^{m_i-1}r^i(n^j\alpha_{i,j})\]
    که
    $\alpha_{i,j}$
    اعداد ثابت هستند.

\end{THEOREM}

\begin{PROBLEM}
    نمونه سوال اضافه شود
\end{PROBLEM}
