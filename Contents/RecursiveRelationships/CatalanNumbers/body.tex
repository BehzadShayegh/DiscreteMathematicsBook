\SECTION{اعداد کاتالان}
اعداد کاتالان دنباله‌ای از اعداد طبیعی هستند که در مسائل شمارش متنوعی ظاهر می‌شوند.

\SUBSECTION{روابط اعداد کاتالان}
\subfile{./examples/1.tex}
\begin{DEFINITION}
      به اعدادی که در رابطه بازگشتی
    \[\begin{cases}
        C_{n+1}=\sum_{i=0}^n C_iC_{n-i} & n\geq 0 \\
        
        C_0=1
    \end{cases}
    \]
    صدق می کنند,
    \FOCUSEDON{ اعداد کاتالان}
      می‌گوییم..
\end{DEFINITION}
حال می‌خواهیم فرم غیر بازگشتی این دنباله را بیابیم.
\subfile{./examples/2.tex}
\begin{THEOREM}
    رابطه غیر بازگشتی اعداد کاتالان به صورت
    \[C_n=\binom{2n}{n}-\binom{2n}{n+1}=\dfrac{1}{n+1}\binom{2n}{n}\]
    است
    .
\end{THEOREM}

همانطور که در ابتدای بخش گفته شد, کاربرد این دنباله در حل مسائل شمارش است. در ادامه تعدادی از این مسائل را می‌بینیم.
\subfile{./examples/3.tex}
\subfile{./examples/4.tex}