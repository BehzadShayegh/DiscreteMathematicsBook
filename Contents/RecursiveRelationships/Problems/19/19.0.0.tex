\p
    این مساله را با حالت‌بندی روی جعبه اول حل می‌کنیم. در جعبه اول می‌توان حداکثر \( \lfloor \frac{n}{k} \rfloor \) جنس قرار داد زیرا اگر از این مقدار بیشتر شود مجبوریم در یک جعبه دیگر کمتر از این مقدار قرار دهیم که ممکن نیست. بنابراین رابطه‌ی نهایی برای \( A_{n,k} \) که تعداد حالات تقسیم n کالا در k جعبه است یک سیگما روی این حالات جعبه اول است که می‌شود:
\[ A_{n,k} = \sum\limits_{i=1}^{\lfloor \frac{n}{k} \rfloor}  A_{n-ik,k-1} \]در جواب بالا، تفریق مقدار \lr{ik} در اندیس به این خاطر است که هر تعداد وسیله که در جعبه اول قرار دهیم، باید در جعبه‌های دیگر نیز حداقل به آن تعداد وسیله قرار دهیم و بنابراین برای i جعبه، تعداد ik وسیله ابتدا باید در هر جعبه به صورت ثابت قرار بگیرد و امکان حالت‌بندی ندارد. 

به دلیل مطرح شدن قید وجود حداقل یک شی در هر جعبه، اندیس شروع ۱ است.
