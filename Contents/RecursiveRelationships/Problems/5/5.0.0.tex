	\p
تابع مولد
$G(x)$
را متناظر با دنباله‌ی بازگشتی گفته شده در نظر می‌گیریم:
$$G(x) = \sum\limits_{n=0}^{\infty}a_n.x^n$$
طرفین رابطه را در 
$x^{n+2}$
ضرب می‌کنیم و سیگما می‌گیریم:
$$\sum\limits_{n=0}^{\infty}a_{n+2}.x^{n+2} = 3\sum\limits_{n=0}^{\infty}a_{n+1}.x^{n+2} - 2\sum\limits_{n=0}^{\infty}a_n.x^{n+2}$$
که معادل است با:
$$G(x) - x = 3x.G(x) - 2x^2.G(x) = $$
$$\frac{x}{1-3x+2x^2}$$
با استفاده از تجزیه کسر داریم:
$$G(x) = \frac{x}{1-3x+2x^2} = \frac{A}{x-1} + \frac{B}{2x-1}$$
با حل دو معادله دو مجهول به‌دست می‌آید:
$$G(x) = \frac{-1}{1-x} + \frac{1}{1-2x}$$
که کسرهای سمت راست توابع مولد آشنایی هستند:
$$G(x) = -\sum\limits_{n=0}^{\infty}x^n + \sum\limits_{n=0}^{\infty}2^n.x^n = \sum\limits_{n=0}^{\infty}(2^n - 1).x^n$$
بنابراین:
$$a_n = 2^n - 1$$
