\p
            ایده حل این سوال شبیه سوال ۳ است. ابتدا مساله را مثل شکل زیر مدل می کنیم. در این شکل، یک جدول دیده می‌شود که در آن، هر خانه یک وضعیت مکانی مشخص را برای ریک تعقیب کننده و ریک تعقیب شونده نشان می دهد. در این جدول ستون‌ها از سمت چپ از صفر تا n شماره گذاری می شوند و ستون i ام نشان دهنده وضعیت‌های ممکن بعد از گذشت i ثانیه است که ریک تعقیب شونده در خانه به طول i در مسیر قرار دارد و ریک تعقیب کننده می‌تواند در هر خانه ای از طول ۰ تا i حضور داشته باشد. بنابراین اگر در شکل زیر خانه سمت چپ را مبدا مختصات در نظر بگیریم، خانه با مختصات \lr{(3, 2)} نشان می‌دهد که ریک تعقیب شونده در خانه سوم و ریک تعقیب کننده در خانه دوم است. در این جدول، خانه‌های قطر \lr{(i, i)} به معنی رسیدن دو ریک در نقطه i ام به هم دیگر است. حال ما مساله را با حالت بندی روی اولین نقطه ای که بعد از حرکت دو ریک دوباره با هم ملاقات می‌کنند حل می‌کنیم. فرض کنید در مثال شکل زیر این نقطه همان خانه سبز باشد که اولین بار در آن ریک تعقیب کننده به ریک تعقیب شونده رسیده است. زمانی که این اتفاق می‌افتد می‌دانیم که هیچ کدام از خانه‌های قبلی روی قطر در مسیر نبوده اند بجز خانه \lr{(0, 0)} که نقطه شروع است. بنابراین مسیر ریک‌ها تا خانه سبز‌ تماما از داخل خانه‌های زرد بوده است. اگر تعداد حالات حرکت دو ریک را در صورتی که بدانیم حتما در بازه 1 تا n برخورد داشته اند، برای جدول n تایی، \( C_n \) در نظر بگیریم، ملاقات دو ریک در خانه \lr{(i, i)} برای اولین بار، مساله را به دو زیرمساله تقسیم می‌کند. یک زیر مساله بعد از خانه ملاقات که با رنگ بنفش در مثال مشخص شده است و در حالت کلی برابر است با \( C_{n-i} \) و یک زیر مساله قبل از خانه ملاقات که با رنگ زرد مشخص شده است و برابر است با \( C_{i-1} \)، زیرا بعد از حرکت اجباری اول از خانه \lr{(0, 0)} به خانه \lr{(1, 0)} دوباره باید یک جدول زیرمساله برای طول \lr{i-1} را حل کنیم. بنابراین همانند سوال ۳ به یک رابطه بازگشتی بر اساس نقطه اولین ملاقات بعد از شروع می‌رسیم که برابر است با:
            \[ C_n = \sum\limits_{i=1}^{n} C_{i-1}C_{n-i} \]بنابراین دوباره به رابطه اعداد کاتالان رسیدیم و طبق فرمول بسته برای جواب این معادله داریم:
            \[ C_n = \frac{(2n)!}{(n+1)!n!} \]حال نکته ای که در مورد این جواب وجود دارد این است که در بدست آوردن آن فرض شده حتما یک بار دو ریک در نقاط 1 تا i به هم می‌رسند، در حالی که چنین چیزی لزوما درست نیست. برای حل این مشکل، یک ستون فرضی در انتهای شکل در نظر می‌گیریم که در واقع مثل این است که از جدول \( C_{n+1} \) استفاده می‌کنیم و می‌گوییم یک حرکت اجباری در انتها از هر خانه در ستون n ام به خانه \lr{(n+1, n+1)} انجام می‌شود تا طی این حرکت تضمین کنیم حتما در طی \( n+1 \) حرکت برخورد صورت گرفته و می‌توانیم جدول حاصل را از راه بالا حل کنیم. از آن جایی که از هر خانه در ستون n ام، تنها یک حرکت ممکن به خانه \lr{(n+1, n+1)} وجود دارد بنابراین اضافه کردن این ستون در تعداد حالات تفاوتی نیز ایجاد نمی‌کند. بنابراین به این نتیجه می‌رسیم که برای مسیر به طول n جواب مساله برابر است با \( C_{n+1} \) پس در صورت تعریف جواب مساله به عنوان \( G_n \) داریم:
            \[ G_n = C_{n+1} = \frac{(2n+2)!}{(n+2)!(n+1)!} \]
            \p 
            \centerimage{0.2}{figure5.png}