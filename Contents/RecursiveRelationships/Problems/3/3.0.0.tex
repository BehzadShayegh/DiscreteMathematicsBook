\p
    نام عروسک روسی چاقی را که باید $2n$ عروسک روسی چاق در آن بگذاریم، عروسک مادر می‌گذاریم. 
    
    فرض کنید تعداد روش‌های گذاشتن $2n$ عروسک روسی چاق داخل عروسک مادر، $a_n$ باشد.
    
    برای $n=0$ فقط 1 حالت وجود دارد، عروسک مادر را خالی می‌گذاریم، پس $a_0=1$.
    
    برای $n=1$ فقط 1 حالت وجود دارد، دو عروسک روسی چاق را داخل عروسک مادر می‌گذاریم، پس $a_1=1$.
    
    برای $n>1$ ابتدا 2 عروسک روسی چاق را داخل عروسک مادر می‌گذاریم سپس $2k$ عروسک روسی چاق را به $a_k$ حالت داخل عروسک روسی چاق سمت چپ و $2n-2k-2$ عروسک روسی چاق را به $a_{n-k-1}$  حالت داخل عروسک روسی چاق سمت راست می‌گذاریم. 
    
    پس به رابطه بازگشتی زیر می‌رسیم:
    
    $$\begin{cases}
        a_n = \sum\limits_{k=0}^{n-1} a_{k}  a_{n-k-1} 
        
        
        a_0 = 1
    \end{cases}$$
    
    که همان رابطه معروف برای اعداد کاتالان است و جواب ساده شده آن به فرم 
    $C_n = \frac{1}{n+1}\binom{2n}{n}$
    است.
