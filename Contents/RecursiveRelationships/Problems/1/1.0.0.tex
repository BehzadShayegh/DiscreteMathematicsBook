
\p
    برای حل این سوال از دنباله کمکی $y$ استفاده می‌کنیم که در آن $y_n$ بیانگر تعداد کلمات $n$ حرفی با شرایط صورت سوال است که با $a$ آغاز می‌شوند. همچنین طبق اصل تقارن می‌توان گفت، تعداد کلمات $n$ حرفی با شرایط صورت سوال که با $b$ آغاز می‌شوند هم برابر $y_n$ است.
    
    به همین ترتیب می‌توان دریافت، تعداد کلماتی که با حرف $c$ آغاز می‌شوند برابر با $x_{n-1}$ است. بنابراین داریم:
    
    $$x_n=2y_n+x_{n-1}$$
    
    حال کلمه‌ای $n$ حرفی که با حرف $a$ آغاز می‌شود را در نظر می‌گیریم. حرف دوم این کلمه ممکن است برابر $b$ یا $c$ باشد. اگر حرف دوم برابر $b$ باشد آنگاه بقیه حروف را به $y_{n-1}$ روش می‌توان پر کرد. در صورتی که حرف دوم برابر $c$ باشد آنگاه بقیه حروف را به $x_{n-2}$ روش پر می‌شوند. بنابراین داریم:
    
    $$y_n=y_{n-1}+x_{n-2}$$
    
    با استفاده از این معادلات می‌توان گفت:
    
    $$x_n=2y_n+x_{n-1}$$
    
    $$x_n=2(y_{n-1}+x_{n-2})+x_{n-1}$$
    
    $$x_n=(2y_{n-1}+x_{n-2})+x_{n-2}+x_{n-1}$$
    
    $$x_n=x_{n-1}+x_{n-2}+x_{n-1}$$
    
    $$x_n=2x_{n-1}+x_{n-2}$$
