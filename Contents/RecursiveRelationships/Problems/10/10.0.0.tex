\p
    ماتریس‌های متقارن $n\times n$ با درایه‌های ۰ و ۱ را که در هر سطر آن‌ها دقیقا یک درایه ۱ وجود داشته باشد به دو دسته تقسیم می‌کنیم.

    \begin{enumerate}
        \item
        
        دسته اول متشکل از ماتریس‌هایی است که درایه با مقدار ۱ سطر اول آن‌ها در ستون اول آمده باشد.
        
        چنانچه از هر یک از این ماتریس‌ها سطر اول و ستون اولشان را حذف کنیم(توجه کنید که اگر سطر و ستون $i$ام یک ماتریس متقارن $n\times n$ را حذف کنیم آنگاه یک ماتریس متقارن $(n-1)\times(n-1)$ به دست می‌آید) ماتریس متقارن $(n-1)\times(n-1)$ای به دست می‌آید که در هر سطر آن دقیقا یک درایه ۱ وجود دارد و از آنجایی که با اضافه کردن یک سطر و ستون با درایه‌های 0 بجز در تقاطع آن‌ها به ابتدای هر ماتریس متقارن $(n-1)\times(n-1)$ به ماتریس $n\times n$ متناظر آن می‌رسیم؛ نتیجه می‌گیریم، تعداد ماتریس‌های این دسته برابر $a_{n-1}$ است.
        \item
        
        دسته دوم متشکل از ماتریس‌هایی است که درایه با مقدار ۱ سطر اول آن‌ها در ستونی غیر از ستون اول آمده باشد.
        
        یکی از این ماتریس‌ها را در نظر بگیرید و فرض کنید درایه با مقدار ۱ سطر اول آن در ستون $k$ام قرار دارد. چنانچه سطر‌های اول و $k$ام این ماتریس و ستون‌های اول و $k$ام این ماتریس ‌را حذف کنیم ماتریس متقارن $(n-2)\times(n-2)$ به دست می‌آید که در هر سطر آن دقیقا یک درایه ۱ وجود دارد. از آنجایی که مقدار $k$،$n-1$ حالت دارد و مانند حالت قبل این عملیات برگشت پذیر است(به $n-1$ حالت مختلف) نتیجه می‌گیریم؛ تعداد ماتریس‌های این دسته برابر $(n-1)\times a_{n-2}$ است.
        
        با جمع زدن این دو حالت به رابطه بازگشتی $a_n=a_{n-1}+(n-1)\times a_{n-2}$ خواهیم رسید.
    \end{enumerate}
