\p
    قبل از برگزاری یک جلسه به هر شخص شماره صندلی‌ای که باید روی آن بنشیند داده شده است. اما در روز برگزاری جلسه، شخصی که قرار بوده روی آخرین صندلی بنشیند زودتر از همه می‌آید و روی اولین صندلی (صندلی شماره 1) می‌نشیند! 
    
    پس از آن هر شخصی که وارد جلسه می‌شود اگر صندلی مربوط به خود را پر ببیند روی اولین صندلی خالی بعدی (با شماره بیشتر) می‌نشیند. 
    
    الف) ثابت کنید برای هر $n \in N$، با هر ترتیبی که افراد وارد جلسه شوند، همه صندلی‌ای برای نشستن پیدا می‌کنند.
    
    ب) در یک جلسه $n$ نفری به ازای $n \geq 2$، با ترتیب‌های مختلف ورود افراد، پس از نشستن همه، چند حالت مختلف نشستن بوجود می‌آید؟

