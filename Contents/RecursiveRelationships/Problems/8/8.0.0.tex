\p
    افراد حاضر در یک جلسه $n$ نفری را با $A_1, A_2, ... , A_n$ نشان می‌دهیم (شخص $A_i$ قرار بوده روی صندلی $i$ ام بنشیند). همچنین طبق صورت مسئله شخص $A_n$ روی صندلی شماره 1 می‌نشیند. حال به اثبات مسئله می‌پردازیم:
    
    الف) 
    
    برهان خلف: فرض کنید جلسه ای $n$ نفری با ترتیبی از ورود افراد به جلسه و  $1\leq i \leq n-1$ وجود داشته باشد که در آن شخص $A_i$ صندلی ای برای نشستن پیدا نکند. 
    
    چون به تعداد همه افراد، صندلی وجود دارد پس هنگام ورود شخص $A_i$، حتماً یک صندلی خالی وجود دارد. فرض کنید صندلی خالی با بزرگترین شماره، صندلی $j$ام باشد. چون $A_i$ جایی برای نشستن نیافته، پس $i > j$. 
    
    همچنین چون صندلی شماره $j$ خالی است، هیچ کدام از افرادی که شماره صندلیشان از $j$ کوچکتر است روی صندلی های با شماره بیشتر از $j$ ننشسته اند (چون افراد روی اولین صندلی خالی می‌نشینند). 
    
    پس روی صندلی‌های $j+1$ تا $n$ افراد $A_{j+1}$ تا $A_{n-1}$ نشسته اند. اما چون تعداد این افراد از تعداد صندلی‌ها کمتر است، پس همه صندلی‌ها پر نشده‌اند و حداقل یک صندلی خالی در این بین وجود دارد که این با بزرگترین بودن $j$ در تناقض است. 
    
    از تناقض حاصل، حکم اثبات می‌شود.
    
    ب)
    
    فرض کنید برای یک جلسه $n$ نفری تعداد حالات نشستن افراد برابر $x_n$ باشد. 
    
    طبق صورت مسئله، چون هر شخص روی صندلی خودش یا یکی از صندلی‌های بعدی می‌نشیند، پس روی صندلی شماره 2 یکی از اشخاص $A_1$ یا $A_2$ می‌نشیند.  
    
    \begin{enumerate}
        \item
        شخص $A_2$ روی صندلی دوم بنشیند: در این صورت اگر شخص و صندلی دوم را از جلسه حذف کنیم و برای هر $3 \leq i \leq n$ یکی از شماره فرد $i$ام کم کنیم یک جلسه $n-1$ نفری با شرایط مسئله خواهیم داشت. پس تعداد حالات نشستن افراد $x_{n-1}$ است.
        \item
        شخص $A_1$ روی صندلی دوم بنشیند: در این صورت اگر افراد اول و آخر که روی دو صندلی اول هستند را جابجا کنیم و بعد شخص و صندلی اول را از جلسه حذف کنیم و برای هر $1 \leq i \leq n$ یکی از شماره فرد $i$ام کم کنیم، یک جلسه $n-1$ نفری با شرایط مسئله خواهیم داشت. پس تعداد حالات نشستن افراد $x_{n-1}$ است.
    \end{enumerate}
    با جمع زدن این دو حالت به جواب می‌رسیم.
    $$x_n = 2x_{n-1}$$
    و همچنین برای $n=2$، فقط یک روش برای نشستن وجود دارد پس $.x_2 =1$
    
    در نتیجه :
    $$x_n=2^{n-2}$$
