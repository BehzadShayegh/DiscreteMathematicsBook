\begin{PROBLEM}[ادیت شود]
    \p
    اژدهای مورتی که از فرمان های او خسته شده‌است تصمیم می گیرد که فرار کند، پس مورتی ریک را وادار می‌کند که به او در یافتن اژدها کمک کند. ریک که از خواسته های مورتی خسته شده بود 2 دستگاه «میسیکس» به مورتی می‌دهد، هر دستگاه یک میسیکس تولید می‌کند که هدف آن کمک به تولید کننده میسیکس است. مورتی از دستگاه اول k و از دستگاه دوم n میسیکس درست می‌کند. میسیکس‌های دسته اول آبی بوده و با هم همکاری می‌کنند، هر کدام از آنها روی یک خط صاف شروع به گشتن  به دنبال اژدها می‌کنند به این صورت که تمام این خطوط موازی‌اند، اما میسیکس های دسته دوم که همه قرمزاند هر کدام به صورت تکی به دنبال اژدها می‌گردند به این صورت که روی یک خط راست حرکت می‌کنند که هیچ 2 تایی موازی نیستیند و هیچ 3 تایی نیز از یک نقطه نمی گذرند(با خطوط میسیکس‌های آبی هم همین شرایط را دارند) رابطه بازگشتی برای تعداد نواحیی که ایجاد می‌کنند یافته و حل کنید.(امتیازی، پاسخ «این سوال را نمی دانم» برای این سوال قابل قبول نیست.)

    \SOLUTION{
        \p
        تابع \(f_{n}\) را تعریف می‌کنیم تعداد ناحیه‌های تولید شده توسط \lr{n} خط که هیچ سه تایی در یک نقطه برخورد ندارند و هیچ دو تایی از آن‌ها با هم موازی نیستند. در این صورت برای \(f_{n}\) به صورت بازگشتی داریم:
        \[ f_{n} = f_{n-1} + n , f_1 = 2 \]
        که این رابطه از طریق استقرا اثبات می‌شود.
        
        پایه‌ی استقرا: \(f_{1} = 2\) و \(f_{2} = 4 = f_{1} + 2\) 
        
        فرض استقرا: \(f_{n} = f_{n-1} + n\) 
        
        حکم استقرا: \(f_{n+1} = f_{n} + n + 1\)
        
        اثبات: وقتی خط \(n+1\) ام در صفحه قرار می‌گیرد هر \lr{n} تا خط قبلی را قطع می‌کند. نام این خطوط تقاطع را از 1 تا n می‌گذاریم. به این صورت این خط جدید بین هر دو خط متوالی k و \(k+1\) در مسیرش یک ناحیه‌ی جدید درست می‌کند و همینطور، قبل از خط 1 و بعد از خط n نیز هرکدام یک ناحیه‌ی جدید اضافه می‌شود. بنابراین در مجموع \(n+1\) ناحیه جدید به وجود آمده است و رابطه‌ی بالا اثبات می‌شود. 
        
        از حل این رابطه بدست می‌آید:
        \[ f_{n} = n + (n-1) + (n-2) + ... + 2 + f_{1} = \frac{(n+1)n}{2}+1 \]
        حال تابع \(G_{n, k}\) را جواب مساله در نظر می‌گیریم. رابطه‌ی زیر برای این تابع به صورت بازگشتی برقرار است:
        \[ G_{n, k} = G_{n, k-1} + n + 1 \]
        زیرا هر خط جدید از میسیکس‌های موازی آبی که اضافه شود طبق فرض سوال با هیچ کدام از n خط قرمز موازی نیست و طبق اثبات بالا از تلاقی آن با خطوط قرمز \(n+1\) ناحیه جدید تولید می‌شود و چون خطوط آبی با هم موازی اند و برخوردی با هم ندارند تعداد k تاثیری در تعداد نواحی جدید بوجود آمده ندارد. بنابراین از حل معادله‌ی بالا برای k داریم:
        \[ G_{n, k} = G_{n, 1} + (n+1)(k-1) \] که در آن \( G_{n, 1} = f_{n+1} \) زیرا هیچ کدام از میسیکس‌ها با هم موازی نیستند و حضور میسیکس آبی در این سناریو تفاوتی با یک میسیکس قرمز نمی‌کند. بنابراین جواب نهایی این مساله می‌شود:
        \[ G_{n, k} = \frac{n^2 + 3n + 4}{2} + (n+1)(k-1) \]
    }
\end{PROBLEM}