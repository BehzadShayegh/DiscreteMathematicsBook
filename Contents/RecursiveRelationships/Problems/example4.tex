\begin{PROBLEM}[شک دارم جاش اینجا مناسب هست یا نه؟  خط آخر رفت تو نوار کناری]
    \p 
    در یک صفحه تعدادی دایره داریم که
     همه با هم متقاطع هستند و مرز هیچ $ 3 $تایی از یک نقطه نمی‌گذرد. رابطه‌ای بازگشتی برای تعداد نواحی حاصل از این دایره‌ها بر حسب تعداد دایره‌ها بیابید و حل کنید.
     
    برای مثال تعداد ناحیه‌های 2 دایره متقاطع 4 تاست، ۱ ناحیه که در اشتراک هر دوست، ۱ ناحیه‌ای که در هیچ کدام نیست و 2 ناحیه که فقط در یکی است.
    \SOLUTION{
        \p
        لم: هر دایره‌ی جدید با 
        $n$ 
        دایره‌ی قبل
        $2n$ 
        ناحیه‌ی جدید ایجاد می‌کند.
        

        فرمول اویلر برای گراف‌های مسطح: \( v-e+f=2 \) که در آن v تعداد رئوس گراف، e تعداد یال‌ها و f تعداد نواحی ایجاد شده توسط گراف مسطح در صفحه است. 
        

         اثبات لم ابتدای سوال با استفاده از فرمول اویلر برای گراف‌های مسطح انجام می‌شود به این صورت که نقاط تلاقی دایره‌ها را رئوس گراف مسطح و کمان‌های بوجود آمده را یال‌های گراف در نظر می‌گیریم. در این صورت با اضافه کردن هر دایره‌ی جدید به گراف قبلی 
         $2n$
          نقطه‌ی تلاقی جدید بوجود می‌آید و این نقاط دایره‌ی اضافه شده را به
         $2n$
          کمان تقسیم می‌کند. بنابراین در نهایت در صفحه یک گراف مسطح خواهیم داشت که $(n+1)2n$ یال و $(n+1)n$ راس دارد. در این صورت تعداد نواحی تشکیل شده در صفحه از فرمول اویلر قابل محاسبه است:
        \[ f = e - v + 2 = (n+1)2n - (n+1)n + 2 = (n+1)n + 2 \]که برای \(n+1\) دایره است و تعداد نواحی بوجود آمده توسط n دایره می‌شود:
        \[n(n-1)+2 = n^2 - n + 2\]
        همچنین این سوال از راه زیر نیز قابل حل است:
        $$ C_{n+1} = C_n + 2n , C_1 = 2 \Rightarrow C_n - C_1 = \sum\limits_{k=2}^{n} 2(k-2) = n^2-n \Rightarrow C_n = n^2-n+2 $$
       
    }
\end{PROBLEM}