\begin{PROBLEM}
    \p
    با استفاده از توابع مولد نمایی پاسخ صریح رابطه بازگشتی
    $$a_{n+1} = (n+1)(a_n-n+1)$$
    و
    $a_0 = 1 , n \geq 0$
    بدست آورید.
    \SOLUTION{
        \p
        تابع مولد نمایی $G(x)$ را به شکل زیر تعریف می‌کنیم:
        
        $$G(x) = \sum_{n = 0}^{\infty} a_n \frac{x^n}{n!}$$

        طرفین رابطه‌ی بازگشتی را در $\frac{x^{n+1}}{(n+1)!}$ ضرب می‌کنیم و سیگما می‌گیریم:
        
        از آنجایی که
        $ n  \geq 0 $
        پس حدود سیگما را از صفر تا بی‌نهایت قرار می‌دهیم.
        
        \begin{center}
            \medbreak
            $\sum_{n = 0}^{\infty} a_{n+1}\frac{x^{n+1}}{(n+1)!} = \sum_{n = 0}^{\infty} a_n\frac{x^{n+1}}{n!} - \sum_{n = 0}^{\infty} (n-1)\frac{x^{n+1}}{n!}$ 
            \medbreak
            $\sum_{n = 1}^{\infty} a_{n}\frac{x^n}{n!} = x\sum_{n = 0}^{\infty} a_n\frac{x^n}{n!} - \sum_{n = 0}^{\infty} n\frac{x^{n+1}}{n!} + \sum_{n = 0}^{\infty} \frac{x^{n+1}}{n!}$ 
            \medbreak
            $a_0 = 1 $
            و
            $G(x) - a_0 = xG(x) - x^2e^x + xe^x$
            \medbreak
            $G(x) = \frac{1}{1-x} + xe^x$
            \medbreak
            $G(x) = \sum_{n = 0}^{\infty} x^{n} + \sum_{n = 0}^{\infty} \frac{x^{n+1}}{n!}$
        \end{center}
       از آنجایی که تابع مولد نمایی نوشتیم، به دنبال ضریب جمله
       $\frac{x^n}{n!}$
       باید بگردیم. در
       $\sum_{n = 0}^{\infty} x^{n}$
       ضریب 
       $\frac{x^n}{n!}$
       برابر
       $n!$
        است و همین طور ضریب برای طرف دوم 
        نیز
        $n$
        است.
        پس
        $a_n = n! + n$
    }
\end{PROBLEM}