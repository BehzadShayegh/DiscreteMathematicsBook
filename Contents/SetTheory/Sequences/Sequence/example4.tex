\begin{PROBLEM}[جمله عمومی دنباله حسابی]
	\p
   اگر جمله‌ی نخست دنباله‌ای حسابی برابر با
$a_1$
 و قدرنسبت آن برابر با
$d$
باشد، جمله‌ی عمومی این دنباله را حساب کنید.

    \SOLUTION{
    \p
    دنباله به صورت زیر می‌باشد:
    $$a_1, a_2, a_3, \cdots, a_{n-2}, a_{n-1}, a_n$$
    $$a_1, a_1 + d, a_1 + 2d, \cdots, a_1 + (n - 1)d$$
    بنابراین جمله‌ی عمومی به صورت زیر می‌باشد:
    $$a_n = a_1 + (n - 1)d$$
        }
\end{PROBLEM}
    \begin{THEOREM}
        \p
        جمله‌ی عمومی دنباله‌ی حسابی با
        جمله اول $a_1$
        و قدرنسبت $d$ برابر است با:
        $$a_n = a_1 + (n - 1)d$$
    \end{THEOREM}





\begin{PROBLEM}[مجموع جملات دنباله حسابی]
	\p
   اگر جمله‌ی نخست دنباله‌ای حسابی برابر با
    $a_1$
    و قدرنسبت آن برابر با
    $d$
    باشد، مجموع
    $n$
    جمله‌ی نخست این دنباله را حساب کنید.

    \SOLUTION{
        \p
        مجموع
        $n$
        جمله‌ی نخست دنباله عبارت است از:
        $$S_n = a_1 + a_2 + \cdots + a_n$$
        $$S_n = a_1 + (a_1 + d) + (a_1 + 2d) + \cdots + (a_1 + (n - 1)d)$$
        و اگر از سمت دیگر به جملات نگاه کنیم:
        $$S_n = a_n + (a_n - d) + \cdots (a_n - (n - 1)d)$$
        با جمع دو عبارت فوق داریم:
        $$2S_n = n(a_1 + a_n)$$
        $$S_n = \frac{n(a_1+a_n)}{2}$$
        $$S_n = \frac{n(2a_1 + (n-1)d)}{2}$$
            }
\end{PROBLEM}
        \begin{THEOREM}
            \p
            مجموع
            $n$
            جمله‌ی نخست دنباله‌ی حسابی با
            جمله اول $a_1$
            و قدرنسبت $d$
            برابر است با:
            $$S_n = \frac{n}{2}(2a_1 + (n - 1)d)$$
        \end{THEOREM}