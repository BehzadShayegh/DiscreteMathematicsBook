\SUBSECTION{دنباله‌های خاص}

\begin{DEFINITION}
  \p
  \FOCUSEDON{دنباله ثابت}
  به دنباله‌ای گفته می‌شود که مقادیر تمام جملات آن یکسان باشد.
\end{DEFINITION}

\p
به زبان ساده‌تر، اگر
$k$
عدد ثابت دلخواهی باشد، آنگاه دنباله ثابت 
$k$
که به ازای هر 
$n$
با 
$a_n = k$
تعریف شده است، 
یک دنباله ثابت است.
مثال:
$$3, 3, 3, ...$$

\begin{DEFINITION}
    \p
    \FOCUSEDON{دنباله حسابی}
    یا تصاعد حسابی به دنباله‌ای از اعداد گفته می‌شود که اختلاف هر دو جمله‌ی متوالی آن مقداری ثابت باشد. به این مقدار ثابت 
   \FOCUSEDON{قدر}
   \FOCUSEDON{نسبت}
   \FOCUSEDON{دنباله}
   \FOCUSEDON{حسابی}
      گفته می‌شود.
\end{DEFINITION}
	\p
به طور مثال دنباله‌ی زیر، یک دنباله‌ حسابی می‌باشد:
$$2, 5, 8, 11, 14, ...$$

\subfile{./example01.tex}
\subfile{./example1.tex}
\subfile{./example2.tex}

\begin{DEFINITION}
    \p
    \FOCUSEDON{دنباله هندسی}
   دنباله‌ای از اعداد است که نسبت هر دو جمله متوالی آن مقداری ثابت باشد. به این مقدار ثابت 
   \FOCUSEDON{قدر نسبت دنباله هندسی}
  گفته می‌شود.    
\end{DEFINITION}
	\p
به طور مثال دنباله زیر، یک دنباله هندسی می‌باشد:
$$2, 6, 18, 54, ...$$

\subfile{./example02.tex}


