الف)
$$ A(x) = 1 + x + x^2 + ... = \frac{1}{1 - x} $$
$$ A(5x) = 1 + 5x + 25x^2 + ... = \frac{1}{1 - 5x} $$

ب)
$$ A(x) = 1 + x + x^2 + ... = \frac{1}{1 - x} $$ 
$$ A(x^2) = 1 + x^2 + x^4 + ... = \frac{1}{1 - x^2} $$ 
$$ 2A(x^2) = 2 + 2x^2 + 2x^4 + ... = \frac{2}{1 - x^2}$$ 


ج)
$$ A(x) = 1 + x + x^2 + ... = \frac{1}{1 - x} $$ 
$$ A^2(x) = 1 + 2x + 3x^2 + ... = \frac{1}{(1 - x)^2}$$ 
$$ A^2(x) - 1 = 2x + 3x^2 + 4x^3 + ... = \frac{1}{(1 - x)^2} - 1$$
$$ \frac{A^2(x) - 1}{x} = 2 + 3x + 4x^2 + ... = \frac{\frac{1}{(1 - x)^2} - 1}{x}$$

برای تولید دنباله 
$<1, 2, 3, ...>$
می‌توانستیم به جای این که $A(x)$ را در خودش ضرب کنیم، از آن مشتق هم بگیریم.