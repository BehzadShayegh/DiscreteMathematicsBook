\p
این
$17$
نفر را
$17$
راس در نظر می‌گیریم که هر دو راسی از آن‌ها با یک یال به هم وصلند. یال بین
$x$
و
$y$
را آبی می‌کنیم اگر آن‌ها در مورد موضوع
$1$
بحث کنند، یال بین آن دو را سبز می‌کنیم اگر در مورد موضوع
$2$
بحث کنند و آن یال را قرمز می‌کنیم اگر در مورد موضوع
$3$
بحث کنند. پس هر کدام از
$16$
یالی که از
$A$
می‌گذرند، به یکی از رنگ‌های آبی، قرمز یا سبز رنگ شده‌اند. طبق اصل لانه کبوتری حداقل
$6$
راس وجود دارند که با یک رنگ به
$A$
وصل شده‌اند. فرض کنیم این راس‌ها
$B, C, D, E, F, G$
باشند. یعنی یال‌های
$AB, AC, AD, AE, AF, AG$
هم‌رنگ، به طور مثال آبی هستند. این راس‌ها هم با
$15$
یال به هم وصلند. اگر یکی از این یال‌ها آبی باشد، پس حداقل
$3$
نفر وجود دارند که در مورد یک موضوع بحث می‌کنند. فرض کنیم بین این
$5$
راس هیچ یال آبی نباشد، یعنی همه‌ی آن‌ها در مورد موضوع
$2$
یا
$3$
بحث می‌کنند. یک راس مانند
$B$
را در نظر می‌گیریم. از
$B$
،
$5$
یالی که به
$5$
راس دیگر می‌رسد را در نظر می‌گیریم. طبق اصل لانه کبوتری
$3$
تا از این یال‌ها هم‌رنگ هستند. اگر بین این سه راس یک یال قرمز باشد که یک مثلث تشکیل می‌شود و مسئله حل است. در غیر این صورت این سه راس دو به دو با یال سبز به هم وصل هستند که باز هم مثلث تشکیل می‌شود و مسئله اثبات می‌شود.