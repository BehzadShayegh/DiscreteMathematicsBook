\p
فرض می‌کنیم
$S_k$
برابر تعداد دروسی باشد که فرد از روز اول تا روز
$k$
ام خوانده است. پس
$k$
برابر با
$1$
یا
$2$

$\cdots$

$77$
است. چون فرد در هر هفته حداکثر
$12$
درس خوانده است، پس:
$$S_{77} \leq 12 \times 11 = 132$$
هم‌چنین چون فرد در هر روز حداقل یک درس خوانده است، پس:
$$1 \leq S_1 \leq S_2 \leq \cdots \leq S_{77} \leq 132$$
$$\Downarrow$$
$$20 \leq S_1 + 20 \leq S_2 + 20 \leq \cdots \leq S_{77} + 20 \leq 152$$
این
$77$
عدد به همراه
$77$
عدد قبلی
$154$
عدد بین
$1$
تا
$152$
هستند. پس حداقل یکی از این دو عدد با هم برابرند ولی هیچ کدام از
$S_i$
ها یا
$S_i + 20$
ها با هم برابر نیستند. پس یکی از
$S_i$
ها با
$S_j + 20$
برابر است. در نتیجه
$i$
و
$j$
ای وجود دارد به طوری که
$S_i - S_j = 20$
باشد. این تفاضل همان تعداد
(ادامه جواب سوال در پاسخ‌نامه وجود نداشت)