        \p
طبق دوگانه‌شماری مسئله را به این تساوی مربوط می‌کنیم. تعداد گراف‌های برچسب‌دار
$n$
راسی دارای فقط
$2$
یال برابر است با:
$$\binom{\binom{n}{2}}{2}$$
سمت راست تساوی را به صورت زیر می‌شماریم:
\begin{enumerate}
\item
تعداد گراف‌های برچسب‌دار
$n$
راسی دارای فقط
$2$
یال با راس مشترک. هر
$3$
راسی از گراف را که انتخاب کنیم، به
$3$
طریق می‌توانیم از داخل آن
$2$
یال با راس مشترک بیرون بکشیم که تعداد حالات آن برابر است با:
$$3\binom{n}{3}$$
\item
تعداد گراف‌های برچسب‌دار
$n$
راسی دارای فقط
$2$
یال بدون راس مشترک. هر
$4$
راسی را که انتخاب کنیم، به
$3$
طریق می‌توانیم از داخل آن
$2$
یال بدون راس مشترک بیرون بکشیم که تعداد حالات آن برابر است با:
$$3\binom{n}{4}$$
مجموع دو حالات فوق طبق اصل جمع و با کمک اتحاد پاسکال برابر است با:
$$3\binom n 3 + 3\binom n 4 = 3(\binom n 3 + \binom n 4) = 4\binom{n+1}{4}$$
\end{enumerate}